% mnras_template.tex 
%
% LaTeX template for creating an MNRAS paper
%
% v3.0 released 14 May 2015
% (version numbers match those of mnras.cls)
%
% Copyright (C) Royal Astronomical Society 2015
% Authors:
% Keith T. Smith (Royal Astronomical Society)

% Change log
%
% v3.0 May 2015
%    Renamed to match the new package name
%    Version number matches mnras.cls
%    A few minor tweaks to wording
% v1.0 September 2013
%    Beta testing only - never publicly released
%    First version: a simple (ish) template for creating an MNRAS paper

%%%%%%%%%%%%%%%%%%%%%%%%%%%%%%%%%%%%%%%%%%%%%%%%%%
% Basic setup. Most papers should leave these options alone.
\documentclass[fleqn,usenatbib]{mnras}

% MNRAS is set in Times font. If you don't have this installed (most LaTeX
% installations will be fine) or prefer the old Computer Modern fonts, comment
% out the following line
% Depending on your LaTeX fonts installation, you might get better results with one of these:
%\usepackage{mathptmx}
%\usepackage{txfonts}

% Use vector fonts, so it zooms properly in on-screen viewing software
% Don't change these lines unless you know what you are doing
\usepackage[T1]{fontenc}

% Allow "Thomas van Noord" and "Simon de Laguarde" and alike to be sorted by "N" and "L" etc. in the bibliography.
% Write the name in the bibliography as "\VAN{Noord}{Van}{van} Noord, Thomas"
\DeclareRobustCommand{\VAN}[3]{#2}
\let\VANthebibliography\thebibliography
\def\thebibliography{\DeclareRobustCommand{\VAN}[3]{##3}\VANthebibliography}


%%%%% AUTHORS - PLACE YOUR OWN PACKAGES HERE %%%%%

% Only include extra packages if you really need them. Common packages are:
\usepackage{graphicx}	% Including figure files
\usepackage{amsmath}	% Advanced maths commands
\usepackage{amssymb}	% Extra maths symbols
\usepackage{newtxtext,newtxmath}


%%%%%%%%%%%%%%%%%%%%%%%%%%%%%%%%%%%%%%%%%%%%%%%%%%

%%%%% AUTHORS - PLACE YOUR OWN COMMANDS HERE %%%%%

% Packages for slightly better tables
\usepackage{booktabs}
% SI units package
\usepackage{siunitx} % Better units, especially SI
% Units used in work
\DeclareSIUnit[]\solarmass
{\text{\ensuremath{\textup{M}_{\odot}}}}
\DeclareSIUnit[]\solarluminosity
{\text{\ensuremath{\textup{L}_{\odot}}}}
\DeclareSIUnit[]\solarradius
{\text{\ensuremath{\textup{R}_{\odot}}}}
\DeclareSIUnit[]\year
{\text{yr}}
\DeclareSIUnit[]\au
{\text{AU}}
\DeclareSIUnit[]\parsec
{\text{pc}}
\DeclareSIUnit[]\erg
{\text{erg}}
\DeclareSIUnit[]\arcsecond
{\text{as}}

% Author commands
\newcommand{\ts}{\textsuperscript}

% Please keep new commands to a minimum, and use \newcommand not \def to avoid
% overwriting existing commands. Example:
%\newcommand{\pcm}{\,cm$^{-2}$}	% per cm-squared

%%%%%%%%%%%%%%%%%%%%%%%%%%%%%%%%%%%%%%%%%%%%%%%%%%

%%%%%%%%%%%%%%%%%%% TITLE PAGE %%%%%%%%%%%%%%%%%%%

% Title of the paper, and the short title which is used in the headers.
% Keep the title short and informative.
\title[Dust formation simulations in WCd systems]{An exploration of dust formation within WCd systems using an advected scalar dust model}

% The list of authors, and the short list which is used in the headers.
% If you need two or more lines of authors, add an extra line using \newauthor
\author[J. W. Eatson et al.]{
J. W. Eatson\thanks{E-mail: \href{mailto:py13je@leeds.ac.uk}{py13je@leeds.ac.uk}} \&
J. M. Pittard
\\
School of Physics and Astronomy, University of
       Leeds, Woodhouse Lane, Leeds LS2 9JT, UK\\  
}

% These dates will be filled out by the publisher
\date{Accepted XXX. Received YYY; in original form ZZZ}

% Enter the current year, for the copyright statements etc.
\pubyear{2022}

% Don't change these lines
\begin{document}
\label{firstpage}
\pagerange{\pageref{firstpage}--\pageref{lastpage}}
\maketitle

% Abstract of the paper
\begin{abstract}
\noindent
Dust production is one of the more curious phenomena observed in massive binary systems with interacting winds.
The high wind temperatures, UV photon flux and violent shocks should destroy any dust grains that condense, however a handful of systems have been observed producing dust with yields equivalent to 1-10\% of the total mass of the stellar winds.
In order to better understand this phenomenon a parameter space exploration was performed using a series of numerical models of dust producing carbon phase Wolf-Rayet (WCd) systems.
These models incorporated a passive scalar dust model simulating dust growth, destruction and radiative cooling.
Sensible dust yields were produced by these simulations, with significant changes that were not solely due to the changing mass loss rates in the systems.
This variability was found to be strongly related to the strength of thin-shell and Kelvin-Helmholtz (KH) instabilities present.
We find that there was a pronounced increase in the dust production rate due to an increased wind velocity shear between the colliding winds, with a high velocity shear inducing KH instabilities and drastically increasing dust yields.

\end{abstract}

% Select between one and six entries from the list of approved keywords.
% Don't make up new ones.
\begin{keywords}
stars: Wolf-Rayet -- methods: numerical -- binaries: general
\end{keywords}

%%%%%%%%%%%%%%%%%%%%%%%%%%%%%%%%%%%%%%%%%%%%%%%%%%

%%%%%%%%%%%%%%%%% BODY OF PAPER %%%%%%%%%%%%%%%%%%

\section{Introduction}

Binary systems with colliding stellar winds are a fascinating type of system, capable of producing a variety of peculiar phenomena.
The shocks produced from these interacting systems create some of the most luminous persistent stellar-mass X-ray sources in the night sky \citep{rossloweSpatialDistributionGalactic2015}.
Within the wind collision region the available mechanical energy rivals the radiative energy of many stars, producing shocks with temperatures up to $10^8$ \si{\kelvin}.

Despite such high temperatures, in particularly energetic colliding wind binary (CWB) systems, dust has been observed to form.
In particular, dust formation occurs around evolved Wolf-Rayet WC sub-type stars that are partnered with an OB type main sequence star (a WR+OB binary).
\cite{allenInfraredPhotometryNorthern1972} first attributed IR excess around WC systems to dust in the form of amorphous carbon grains; however, the high wind temperatures and extremely high luminosities around WC systems are such that dust grains would be readily destroyed through sublimation processes.
Dust has been observed to form readily in binary systems (so-called WCd systems), despite an additional highly luminous star and shocks that would quickly destroy dust acting upon these nascent, fragile dust grains.
The exact mechanisms of dust formation as well as the evolution of dust within these systems are poorly understood.
However dust formation rates can be extremely high, up to $10^{-6} \, \si{\solarmass\per\year}$, or up to approximately $36\%$ of the total wind by mass in the case of WR104 \citep{lauRevisitingImpactDust2020}.

% Persistent and episodic dust forming systems
% Discuss leading theories briefly
Within different colliding wind binary systems, dust may form either continuously or periodically.
The first such observed dust forming system was the episodic dust forming system WR140, first reported by \cite{williamsMultifrequencyVariationsWolfrayet1990}, who observed a significant and highly variable infrared excess, consistent with emission from dust grains.
The dust production rate was later found to vary by a factor of 40 over an orbit of  \SI{7.9}{\year} \citep{van1999wolf,thomasOrbitStellarMasses2021}.
Persistent dust forming systems were subsequently discovered, such as WR140 by \cite{tuthill_dusty_1999} and WR98a by \cite{monnierPinwheelNebulaWR1999}, which is used as the prototypical system in this paper.
Whilst the exact mechanism for this condition is not currently known, there is a strong correlation between periodicity and eccentricity, with less eccentric systems forming dust continuously, while highly elliptic systems exhibit periodic dust formation
\citep{crowther_dust_2003}.
Due to this orbital dependency, it is likely that there is an optimal dust forming separation, where dust can form in large quantities. This could be due to factors such as strong post shock cooling, which is highly dependent on the wind speed and orbital separation.
Additionally, dust may be protected from the bulk of the stellar radiation due to the extremely large degree of extinction that may occur in the dense post-shock environment of radiative shocks \citep{cherchneffDustFormationCarbonrich2015}.

% Why is it so hard to observe these systems?

Direct observation of dust forming CWBs and in particular the wind collision region (WCR) is exceptionally difficult for a number of reasons:

\begin{itemize}
  \item WR+OB CWB systems are extremely rare. Of the 667 catalogued WR stars at time of writing, 106 have been confirmed to be in a binary system \citep{rossloweSpatialDistributionGalactic2015}.
  \item A WC star is required for dust formation. No Nitrogen sub-type Wolf-Rayet (WN) have been observed to form dust.
  \item Not all WC+OB systems are dust producing, limiting the sample size further.
  \item Overall 56 dust forming systems with a known spectral type have been observed. Despite producing an extremely large quantity of dust in their local region, they are outnumbered by AGB stars by $\sim 3$ orders of magnitude \citep{ishiharaGalacticDistributionsCarbon2011}.
  \item Galactic CWB systems are comparatively distant from earth. For instance, WR 104, a well-studied system, is $\sim \SI{2.5}{\kilo\parsec}$ distant. This prevents observations of these systems at a high angular resolution.
  \item Grain growth from small nucleation grains is predicted to be very rapid in CWB systems. Therefore studying the initial grain evolution would require observational equipment of extremely high angular resolution \citep{zubkoPhysicalModelDust1998a}.
\end{itemize}

For these reasons, numerical simulations are useful for modelling the growth of dust grains within this unresolved region.
% Proposal of work, what is this project covering?
In order to better understand what influences dust production in a CWB system, a parameter space exploration of the wind and orbital parameters was performed.
In particular the orbital separation, mass-loss rate and wind velocity were modified for both stars in order to influence the wind momentum ratio, $\eta$, and the cooling parameter, $\chi$.
The wind momentum ratio is defined as

\begin{equation}
  \eta = \frac{\dot{\text M}_\text{OB} v^\infty_\text{OB}}{\dot{\text M}_\text{WR}v^\infty_\text{WR}} ,
\end{equation}

\noindent
where $\dot{\text{M}}$ is the mass loss rate of a star and $v^\infty$ is the terminal velocity of a star's outflow.
A low value for $\eta$ indicates that the winds are extremely imbalanced, with the WR typically dominating the wind dynamics of the system.
The wind momentum ratio determines for a given orbital separation, $d_\text{sep}$, the distance from each star to the apex of the wind collision.
We define the terms $r_\text{WR}$ and $r_\text{OB}$, representing the distance from the WR and OB stars to the WCR:

\begin{subequations}
  \begin{align}
    r_\text{WR} & = \frac{1}{1+\eta^{1/2}} d_\text{sep} , \\
    r_\text{OB} & = \frac{\eta^{1/2}}{1+\eta^{1/2}} d_\text{sep} .
  \end{align}
\end{subequations}

\noindent
The half-opening angle of the WCR can be estimated by the formulae:

\begin{equation}
  \theta_c \simeq 2.1 \left( 1 - \frac{\eta^{2/5}}{4}\right) \eta^{-1/3} ~~~ \text{for} ~ 10^{-4} \leq \eta \leq 1 ,
\end{equation}

\noindent
to a relatively high degree of accuracy \citep{eichler_particle_1993,pittardCollidingStellarWinds2018}.

The cooling parameter, $\chi$, compares the cooling time to the escape time from the shocked region for a parcel of gas in the immediate post-shock environment. An approximation can be made using the known parameters of a system using the equation:

\begin{equation}
    \chi = \frac{t_\text{cool}}{t_\text{esc}} \approx \frac{v_8^4 d_{12}}{\dot{\text M}_{-7}} , 
\end{equation}

\noindent
where $v_8$ is the wind terminal velocity in units of $10^8$ \si{cm.s^{-1}}, $d_{12}$ is the distance to the WCR apex in units of $10^{12}$ \si{cm}, and $\dot{\text M}_{-7}$ is the mass loss rate in units of $10^{-7} \si{\solarmass\per\year}$ \citep{stevens_colliding_1992}.
$\chi \leq 1$ indicates that radiative cooling is very important, while $\chi \gg 1$ indicates that the system is adiabatic.
Strong cooling is aided with slow, dense winds and a high metallicity.
As such in many systems the post-shock WR flow will rapidly cool from the immediate post-shock temperature of $\sim 10^8 \, \si{\kelvin}$ to temperatures in the dust formation range, $T \lesssim 10^4 \, \si{\kelvin}$.
A strongly radiating WCR can also be significantly compressed far more as it loses energy.
In comparison, an adiabatic WCR is limited to a maximum density increase of a factor of 4 above the pre-shock wind density for $\gamma = 5/3$.
The density increase and cool temperatures result in rapid dust growth and protection from the stellar UV radiation in some systems.

For this paper, we aim to explore how dust formation is affected by the orbital and wind parameters of persistent dust forming WR+OB systems.
This is performed by running a series of hydrodynamical simulations with an advected scalar dust model.
In section \ref{sec:methodology} we outline the methodology of our simulations, and how our dust model is implemented. 
We discuss our model series parameters, and why these parameters were chosen in section \ref{sec:p1-model-parameters}.
Finally discuss our results and conclude in section \ref{sec:p1-results} and section \ref{sec:p1-conclusion}.

\section{Methodology}
\label{sec:methodology}

Numerical simulations within this paper utilise the Athena++ hydrodynamical code, a highly modular modern fluid dynamics code \citep{stoneAthenaAdaptiveMesh2020}.
Simulations are generated in 3D and the Euler hydrodynamical equations are solved in the form:

\begin{subequations}
  \begin{align}
    \frac{\partial\rho}{\partial t}+\nabla \cdot \left(\rho \boldsymbol{u}\right) & = 0 , \\
    \frac{\partial \rho \boldsymbol{u}}{\partial t} + \nabla \cdot \left(\rho \boldsymbol{u} u + P \right) & = 0, \\
    \frac{\partial \rho \varepsilon}{\partial t} + \nabla \cdot \left[ \boldsymbol{u} \left( \rho\varepsilon + P \right) \right] & = \dot E_\text{cool} , 
  \end{align}
\end{subequations}

\noindent
where $\varepsilon$ is the total specific energy ($\varepsilon = \boldsymbol{u}^2/2 + e/\rho $), $\rho$ is the mass density, $e$ is the internal energy density, $P$ is the gas pressure, $\boldsymbol{u}$ is the gas velocity and $\dot E_{cool}$ is the energy loss rate per unit volume from the fluid due to gas and dust cooling, which is elaborated on in section \ref{sec:gas-dust-cooling}.

% Technical details

Athena++ has been configured to run using a piecewise linear reconstruction method with a 4\ts{th} order Strong Stability Preserving Runge-Kutta time-integration method \citep{spiteriNewClassOptimal2002}.
Athena++ was forked from the original repository and additional routines were written for a colliding wind binary scenario.
Routines were created to produce a steady outflow from a small spherical region around a set of cartesian co-ordinates as well as a function to move these co-ordinates with each time-step; these were used to simulate stellar wind outflow and orbital motion, respectively.
Additionally, Athena++ was further modified to include an advected scalar dust model for simulating dust growth and destruction as well as a photon emission cooling model to approximate cooling for gas and dust particles within the fluid.

Athena++ utilises OpenMPI for parallelism, breaking the simulation into blocks, which are distributed between processors.
The block size is variable, but for these simulations a block size of $32\times 32 \times 8$ was found to be optimal.
This meshblock system is also utilised in mesh refinement for increasing effective resolution.
As the CWB systems are being simulated in their entirety, a very large volume needs to be simulated, while at the same time the region between the stars must be resolved with a resolution of at least 100 cells in order to adequately resolve the WCR.
This difference in length scales necessitates the use of static mesh refinement (SMR) to improve the effective resolution of the simulation.
A base coarse resolution of $320 \times 320 \times 40$ cells is defined for the simulations, while a region close to the stars operates at a higher refinement level.
This results in a resolution increase of a factor of $2^{n-1}$ greater than the coarse resolution, where $n$ is the refinement level (see fig. \ref{fig:smr-grid}).
In the case of 7 levels (inclusive of the base, ``coarsest'' level) as used in most of the simulations in this paper, this results in an effective resolution of $20480 \times 20480 \times 2560$ cells.
SMR is utilised instead of Adaptive Mesh Refinement, a more flexible conditional method, as it has proven to be more reliable for our simulations.
As much of the grain evolution occurs a small distance from the WCR stagnation point, much of the simulation volume can be run at a lower resolution without affecting the simulation results.

\begin{figure}
  \centering
  \includegraphics[width=2.5in]{assets/mesh/gridxy.pdf}
  \caption[Static mesh refinement example]{A plot of the blocks used in a 7 level simulation with a block size of $32\times 32 \times 8$ cells. The block density increases dramatically closer to the barycentre. The coarse simulation resolution is $(320\times 320\times 40)$ cells with a block size of $(32\times32\times8)$ cells. The diagram is sliced about the $z$ axis at $z=0$.}
  \label{fig:smr-grid}
\end{figure}

The wind outflow from each star is simulated by replacing the conserved variables (density, momentum and energy) within a small region around the expected position of the stars; this region is typically on the order of 6 maximally refined cells in radius.
This rewrite corresponds to a change in density, $\rho_R$, pressure, $P_R$, and mechanical energy, $E_R$, imparted by an outflowing wind, such that

\begin{subequations}
  \begin{align}
    \rho_R & = \frac{\dot M}{(4 \pi r^2 v_\infty)} , \\
    P_R    & = \rho_R k_B T_w / \mu m_H , \\
    E_R    & = \frac{P_R}{\gamma - 1} + \frac{1}{2} \rho_{R} v_\infty^2 ,
  \end{align}
\end{subequations}

% This may need more explanation, depending on previous equations
\noindent
where $v_\infty$ is the wind velocity as it flows radially from the center of the ``remap zone'', $T_w$ is the wind temperature and $r$ is the radial distance from the current cell to the centre of the remap zone.
Orbits are calculated by moving the remap zones in a manner consistent with Keplerian dynamics, which are repositioned at the start of every timestep.
This orbital speed is also added to the remap wind speed.

% Plasma and dust cooling

\subsection{Gas and dust cooling}
\label{sec:gas-dust-cooling}

Cooling due to photon emission from atoms, ions and free electrons, as well as dust particles, is simulated by removing energy from a cell at each timestep.
The total energy loss is calculated by integrating the energy loss rates due to gas, plasma and dust cooling using the Euler method; in regions with very rapid cooling sub-stepping is used to improve accuracy, with the number of sub-steps being determined by comparing the substep time to the cooling timescale of the cell.
Gas cooling is simulated using a lookup table method.
A data file containing the gas temperature and associated normalised emissivity, $\Lambda_w(T)$, of the wind at that temperature is read into the simulation.
In a typical cooling step, the temperature is calculated and compared with the lookup table to find the closest temperature bins that are lower and higher than the cell temperature.
A linear interpolation is then performed to find an appropriate value for $\Lambda_\text{w} (T)$.
The energy loss in the stellar wind can then be calculated with the formulae:

\begin{equation}
  \frac{dE}{dt} = \left(\frac{\rho}{m_\text{H}}\right)^2 \Lambda_\text{w} (T),
\end{equation}

\noindent
where $\rho$ is the gas density and $m_H$ is the mass of a hydrogen atom.
The lookup table was generated by mixing a series of cooling curves generated by MEKAL simulations of elemental gasses.
These simulations were combined based on the elemental abundances of each wind, with the WC star having typical WC9 abundances and the OB star having a solar abundance (see table \ref{tab:abundances}).
% These are combined based on the elemental abundances of each wind such that:
% \begin{subequations}
%
%   \begin{align}
%     \Lambda_\text{WR}(T) & = n_e n_i \sum{X_\text{WR} \Lambda_\text{WR}{E}(T)}, \\
%     \Lambda_\text{OB}(T) & = n_e n_i \sum{X_\text{OB} \Lambda_\text{OB}{E}(T)},
%   \end{align}
% \end{subequations}
%
% \noindent
% where $n_e$ and $n_i$ are the electron and ion number density, $X$ is the abundance of an element, and $\Lambda_E(T)$ is the cooling parameter of an element.
Figure \ref{fig:cooling-curve} shows the resulting cooling curves used for each star.
%Two lookup tables are used in the simulations, based on the elemental abundances of each star. 
% The Wolf-Rayet star uses a curve with abundances typical of a WC9 star with total hydrogen depletion and a high carbon mass fraction, while the OB star is assumed to have solar abundances.
The most significant abundances used in this projects simulations are presented in table \ref{tab:abundances}.
The cooling regime of the simulations ranges between temperatures of $10^4$ to $10^9\,\si{\kelvin}$.
A floor temperature of $10^4$ \si{\kelvin} is implemented.
Temperatures between $\SI{1e4}{\kelvin} < T \leq \SI{1.1e4}{\kelvin}$ are set to $10^4\,\si{\kelvin}$ as they are assumed to be either rapidly cooling or a part of the stellar wind.

\begin{table}
  \centering
  \begin{tabular}{@{}lll@{}}
  \toprule
  \multicolumn{1}{l}{} & \multicolumn{2}{c}{X(E)} \\ \cmidrule(l){2-3} 
   & Solar & WC9 \\ \midrule
  H & $0.705$ & $0.0$ \\
  He & $0.275$ & $0.546$ \\
  C & $3.07 \times 10^{-3}$ & $0.4$ \\
  N & $1.11 \times 10^{-3}$ & $0.0$ \\
  O & $9.60 \times 10^{-3}$ & $0.05$ \\
  % Ne & $1.75 \times 10^{-3}$ & $0.0$ \\
  % Na & $3.47 \times 10^{-5}$ & $3.47 \times 10^{-5}$ \\
  % Mg & $7.10 \times 10^{-4}$ & $7.10 \times 10^{-4}$ \\
  % Al & $6.13 \times 10^{-5}$ & $6.13 \times 10^{-5}$ \\
  % Si & $8.60 \times 10^{-4}$ & $8.60 \times 10^{-4}$ \\
  % S & $3.82 \times 10^{-4}$ & $3.82 \times 10^{-4}$ \\
  % Ar & $1.01 \times 10^{-4}$ & $1.01 \times 10^{-4}$ \\
  % Ca & $6.15 \times 10^{-5}$ & $6.15 \times 10^{-5}$ \\
  % Fe & $1.52 \times 10^{-3}$ & $1.52 \times 10^{-3}$ \\
  % Ni & $7.65 \times 10^{-5}$ & $7.65 \times 10^{-5}$ \\ \bottomrule
  \hline
  \end{tabular}
  \caption[Abundances by mass used for OB and WR stars]{Abundances used for the OB and WR stars being simulated. Other elements are assumed trace when calculating dust emission \citep{williamsSpectraWC9Stars2015}.}
  \label{tab:abundances}
\end{table}


\begin{figure}
  \centering
  \includegraphics[width=\linewidth]{assets/cooling-curve/cooling-curve-no-elements.pdf}
  \caption[WR and OB $\Lambda(T)$ cooling curves]{Comparison of WC and solar cooling curves for calculating the energy loss due to gas cooling.}
  \label{fig:cooling-curve}
\end{figure}

A model for cooling due to emission from dust grains is also included as dust cooling is expected to play a significant role in each system.
The rate of cooling is calculated using the uncharged grain case of the \cite{dwek_infrared_1981} prescription.
Grains are heated due to collisions with ions and electrons, causing them to radiate, with energy being removed from the simulation.
This assumes that the region being simulated is optically thin to far infrared photons.
The grain heating rate is calculated with the following formulae:

\begin{equation}
    H = 1.26 \times 10^{-19} \frac{n}{A^{1/2}} a^2(\si{\micro\metre}) T^{3/2} h(a,T) , 
\end{equation}

\noindent
where $H$ is the heating rate due to atom and ion collisions, 
$n$ is the particle number density,
$A$ is the mass of the incident particle in AMU,
$a(\si{\micro\metre})$ is the grain radius in microns,
$T$ is the temperature of the ambient gas,
and $h(a,T)$ is the effective grain ``heating factor'', also referred to as the grain transparency. 

To obtain the collisional heating due to incident atoms, $H_\text{coll}$, the heating rates are summated for Hydrogen, Helium, Carbon, Nitrogen and Oxygen atom collisions are summated together:

\begin{equation}
  H_\text{coll} = H_\text{H} + H_\text{He} + H_\text{C} + H_\text{N} + H_\text{O} .
\end{equation}

\noindent
Other elements are not considered as they are present in trivial proportions in both winds.
As a neutral grain is assumed, the grain transparency for each species is calculated with the formulae:

\begin{equation}
  h(a,T) = 1 - \left( 1 + \frac{E_0}{2 k_\text{B} T} \right) e^{- E_0 / k_\text{B} T} ,
\end{equation}

\noindent
where $E_0$ is the initial energy required to overcome the grain's potential and $k_\text{B}$ is the Boltzmann constant.

\begin{figure}
  \centering
  \includegraphics[width=\linewidth]{assets/ionisation-fraction/ionisation-fraction.pdf}
  \caption[OB and WR electron-ion ratios]{A comparison of the electron-ion ratio in both winds as as a function of temperature. Also shown are the electron-to-ion ratios for the individual elements.}
  \label{fig:electron-curve}
\end{figure}

Electron-grain collisional heating, $H_\text{el}$, is modelled using the same calculation for $H_\text{coll}$, albeit with some differences.
One major factor for accurately calculating the energy loss due to electron collisions is that the electron number density, $n_e$, needs to be correct.
This is achieved with a second series of lookup tables that contain the electron-to-ion ratio of each wind across a temperature range of $10^4$ to $10^9\,\si{\kelvin}$ (fig. \ref{fig:electron-curve}).
The electron number density is $n_e = \beta n_i$, where $\beta$ is the electron-to-ion ratio and $n_i$ is the ion number density.
Another difference between calculating electron-grain and gas-grain cooling is calculating electron-grain transparency, which is a significantly more complex problem than calculating ion-grain transparency.
An assumed full opacity proves to be extremely inaccurate at temperatures $>10^6\,\si{\kelvin}$.
Electron-grain transparency is therefore calculated via an approximation described in \cite{dwek_infrared_1981}:

\begin{equation}
  \begin{alignedat}{3}
    h(x^*) & = 1 ,                && ~~ x^* > 4.5, \\
           & = 0.37{x^*}^{0.62} , && ~~ x^* > 1.5 , \\
           & = 0.27{x^*}^{1.50} , && ~~ \text{otherwise,}
  \end{alignedat}
\end{equation}

\noindent
where $x^* = \num{2.71e8} a^{2/3} (\si{\micro\metre})/T$.
This approximation is approximately 4 orders of magnitude faster than using an integration method, while only differing by less than 8\% (fig. \ref{fig:lambdacomparison}).
Grain-grain collisions are not modelled, as this would be difficult to calculate due to the single-fluid model in use.
Further simulations utilising a multi-fluid model could allow for this to be simulated.
Finally, in order to calculate the change in energy due to dust cooling, the rate of energy change, $dE/dt$ is calculated using the formulae:

\begin{subequations}
  \label{eq:transparency-approximation}
  \begin{align}
    \Lambda_\text{d}(T,a) & = \frac{H_\text{coll} + H_\text{el}}{n_\text{H}} , \\
    \frac{dE}{dt}         & = n_\text{T} n_\text{d} \Lambda_\text{d} (T,a) ,
  \end{align}
\end{subequations}

\noindent
where $\Lambda_\text{d}$ is the normalised dust emissivity,
$n_\text{H}$ is the hydrogen number density,
$n_\text{T}$ is the total number density
and $n_\text{d}$ is the dust number density.
The total energy loss rate per unit volume due to gas and dust cooling is given by:

\begin{equation}
	\frac{dE}{dt} = \left( \frac{\rho}{m_\text{H}} \right)^2 \Lambda_\text{w}(T) + n_\text{T} n_\text{d} \Lambda_\text{d} (T,a)
\end{equation}

\begin{figure}
  \centering
  \includegraphics[width=\linewidth]{assets/grain-transparency/lambda-comp.pdf}
  \caption[Comparison of electron transparency methods.]{Dust grain cooling curves, $\Lambda_\text{d}(T,a)$, as a function of temperature for various grain sizes. The estimate method (Eq. \ref{eq:transparency-approximation}, dashed line) is extremely close to the integral value (solid line) aside from at the highest temperatures.}
  \label{fig:lambdacomparison}
\end{figure}

\subsection{Numerical modelling of dust through advected scalars}

The most important modification to Athena++ was the addition of a dust growth and destruction model to simulate the production of dust within the WCR.
A series of passive scalars were used where the dust parameters described by the scalars can evolve and advect through the simulation, analogous to a co-moving fluid, which previous papers have noted is an accurate dynamical model for dust within the WCR \citep{hendrix_pinwheels_2016}.
In these simulations, information about the dust is stored in the form of two variables, the average grain radius, $a$, and the dust-to-gas mass ratio, $z$.
From these constants the dust production rate, number density, and total dust mass can be derived.
A co-moving model allows for a simplified model of dust formation. In such a model, the mean particle velocity between two particles of different size is:

\begin{equation}
  \langle u \rangle = \left[ \frac{8kT}{\pi m_r} \right] ^{1/2} ,
\end{equation}

\noindent
where $m_r$ is the familiar reduced mass between a test particle of mass $m_t$ and a field particle of mass $m_f$:

\begin{equation}
  m_r = \frac{m_f m_t}{m_f + m_t} .
\end{equation}

\noindent
As the dust grain is significantly more massive, the reduced mass is approximately equal to the grain mass, simplifying the dynamics of the simulation in a co-moving case.

Dust growth is modelled through approximating growth due to grain-gas accretion where grains co-moving with a gas perform relatively low-velocity collisions with the surrounding gas, causing it to accrete onto the surface of the dust grain 
\citep{spitzer_jr._physical_2008}.
Assuming a single average grain size the rate of change in the average grain radius is given by

\begin{equation}
  \frac{da}{dt} = \frac{\xi_\text{a} \rho_\text{gr} w_\text{a}}{4 \rho} ,
\end{equation}

\noindent
where $w_a$ is the Maxwell-Boltzmann distribution RMS velocity, $\xi_a$ is the grain sticking efficiency and $\rho_\text{gr}$ is the grain bulk density.
The associated rate of dust density change is found to be

\begin{equation}
  \frac{d\rho_\text{d}}{dt}  = 4 \pi a^2 \rho_\text{g} n_\text{d} \frac{da}{dt} , 
\end{equation}

\noindent
where $\rho_\text{g}$ is the gas density and $n_\text{d}$ is the grain number density.
In this paper we take $\xi_a = 0.1$ as a conservative value, though this value can rise to as high as $1$ in the case of highly charged grains.
A bulk density analogous to amorphous carbon grains of \SI{3}{\gram\per\centi\metre\cubed} is also used.

Dust destruction is calculated via gas-grain sputtering using the \cite{draine_destruction_1979} prescription - a dust grain has a lifespan, $\tau$, which is dependent on the grain radius and the number density of the gas the grain is moving through.
Assuming a spherical grain, the rate of change in mass and radius can be calculated as:

\begin{subequations}
  \begin{align}
    \tau_\text{d} & = 1 \, \text{Myr} \times \frac{a}{n_\text{g}} , \\
    \frac{da}{dt} & = - \frac{a}{\tau_\text{d}} , \\
    \frac{dm}{dt} & = -1.33 \times 10^{-13} a^2 n_\text{g} n_\text{d} \rho_\text{gr} ,
  \end{align}
\end{subequations}

\noindent
where $n_\text{g}$ is the gas number density.
The value for the normalised grain lifespan of \SI{1}{\mega\year\micro\metre\per\centi\metre\cubed} is based on an average lifespan of dust grains in interstellar shocks with shock temperatures between $10^6$ and $\SI{3e8}{\kelvin}$ from \cite{dwekCoolingSputteringInfrared1996}, with additional work by \cite{tielens_physics_1994}.

Application of the dust growth and destruction routines in the code is determined by the gas temperature of a cell.
Dust occurs when $T \leq \SI{1.4e4}{\kelvin}$ whilst dust destruction occurs at temperatures of $T \geq 10^6 \, \si{\kelvin}$.

% //TODO cleanup sentence structures
In order to propagate dust through each simulation, a small initial value for the advected scalars is set in each cell in the remap zones.
An initial grain radius of $a_i = 50 \, \text{\AA}$ and initial dust-to-gas mass ratio of $z_i = 10^{-6}$ is imposed.
Changing $z_i$ does not significantly impact the final dust-to-gas mass ratio of the system as $z$ rapidly increases within the WCR and dust growth in the WCR dominates the total production.
Dust grows to some extent in the unshocked winds due to this propagation method, however it is extremely low compared to dust production within the shocked winds.
A small initial grain radius would also be sensible, as small dust grains are believed to rapidly nucleate from impinging carbon ions 
\citep{harriesThreedimensionalDustRadiativetransfer2004}.
It should also be noted that dust formation around single WC stars has been observed, suggesting that nascent grains are formed within the WC wind, and carried into the WCR as implied by this dust model.

In order to determine if our dust model is producing sensible dust yields, we calculate the maximum expected dust production rate in each system, $\dot{\text{M}}_\text{d,max}$.
This rate would occur if 100\% of the carbon in the WR wind being shocked by the WCR was converted into dust.
%We first define this WCR interaction factor, $f_\text{WR}$:
The fraction of the WR wind that passes through the WCR is given by

\begin{equation}
	f_\text{WR} = \frac{1 - \cos \left(\theta_\text{WR}\right)}{2} ,
\end{equation}

\noindent
where $\theta_\text{WR}$ is the opening angle of the WR shock front, approximated as $\theta_\text{WR} \approx 2 \tan^{-1} ( \eta^{1/3} ) + \pi/9$.
The theoretical maximum dust production rate is then

\begin{equation}
	\dot{\text{M}}_\text{d,max} = \dot{\text{M}}_\text{WR} \text X_\text{C,WR} f_\text{WR},
\end{equation}

\noindent
 where $\text X_\text{C}$ is the carbon mass fraction in the WR star
 \citep{pittardCollidingStellarWinds2018}.


\section{Model Parameters}
\label{sec:p1-model-parameters}

In this paper we do not attempt to model particular systems.
Rather we aim to gain a deeper understanding of the primary influences of dust formation in a CWB system.
A series of simulations were therefore run in order to determine how dust formation varies due to changes in orbital separation and wind momentum ratio.
A baseline simulation with properties similar to WR98a with a circular orbit and identical stellar masses was created.
This baseline simulation has a momentum ratio of $0.02$.
Other simulations were then run with different orbital separations and/or wind momentum ratios.
Another set of simulations were run where the cooling mechanisms were selectively disabled, in order to understand how radiative cooling affects the dust production rate.
Tables \ref{tab:baseline-windproperties} and \ref{tab:baseline-orbits} detail the wind and orbital parameters of the baseline simulation.
The orbital separation is modified by changing the orbital period of the simulation, while the wind momentum ratio is modified by adjusting the mass loss rate and wind terminal velocity for each star.
Two simulation sub-sets for this were performed: simulations where the wind terminal velocities were adjusted for each star and simulations where the mass loss rates for each star were adjusted.

\begin{table}
  \centering
  \begin{tabular}{lll}
  \hline
  Parameter & WR & OB \\ \hline
  $\dot M$ & \SI{5.0e-6}{\solarmass\per\year} & \SI{5.0e-8}{\solarmass\per\year} \\
  $v_\infty$ & \SI{1.0e8}{cm.s^{-1}} & \SI{2.0e8}{cm.s^{-1}} \\
  $T_w$ & \SI{1.0e4}{\kelvin} & \SI{1.0e4}{\kelvin} \\
  \hline
  \end{tabular}
  \caption{Wind properties of the baseline system.}
  \label{tab:baseline-windproperties}
\end{table}

\begin{table}
  \centering
  \begin{tabular}{ll}
  \hline
  $\text{M}_\text{WR/OB}$ & 10.0 \si{\solarmass} \\
  $d_\text{sep}$ & \SI{4.0}{\au} \\
  $P$ & \SI{1.80}{\year} \\
  \hline
  \end{tabular}
  \caption{Baseline system orbital properties.}
  \label{tab:baseline-orbits}
\end{table}

\subsection{Cooling mechanisms}

For this set of simulations, the influence of cooling was changed by varying how cooling works within the simulations.
All simulations in this set keep the same orbital and wind parameters, which are that of the baseline system described in Tables \ref{tab:baseline-windproperties} \& \ref{tab:baseline-orbits}.
One simulation has both plasma and dust cooling in operation (the \texttt{fullcool} simulation), while the other two simulations have plasma cooling only and no cooling, respectively (\texttt{plasmacool} and \texttt{nocool}, Table \ref{tab:cooling-param}).
The final, no radiative cooling simulation instead relies on adiabatic expansion for temperature change in the WCR; as such, this simulation behaves as if it has a $\chi$ value for both winds that is arbitrarily high.
The post-shock flow in the \texttt{nocool} model will also be unable to compress as much due to the lack of energy loss via radiative cooling.
The role of these simulations is to discern whether cooling alone, or other system parameters can affect dust production.

% Discuss why this is important, mention overdensity due to radiative cooling

\begin{table}
  \centering
  \begin{tabular}{lll}
    \hline
    Name & Plasma cooling? & Dust cooling? \\
    \hline
    \texttt{fullcool} & Yes & Yes \\ 
    \texttt{plasmacool} & Yes & No \\
    \texttt{nocool} & No & No \\
    \hline
  \end{tabular}
  \caption{Cooling series simulation parameters.}
  \label{tab:cooling-param}
\end{table}

\subsection{Wind momentum ratio}

Another set of simulations was devised in order to assess the influence of the wind parameters on the formation of dust within a CWB.
As the wind momentum ratio is dependent on both the mass loss rate and wind velocity of each star, each of these properties is modified over a set of different simulations.
$\eta$ is varied from 0.01 to 0.04 by adjusting the wind parameters for each star.
This is further subdivided by which property is modified, either the mass loss rate or wind terminal velocity (Table \ref{tab:vinf-param}).
As the cooling parameter has a much stronger dependency on $v_\infty$ than $\dot{\text{M}}$, the modification of either parameter while maintaining a similar value for $\eta$ allows us to determine whether $\chi$ is the primary parameter determining the formation of dust within WCd systems.
This can be seen when comparing simulations \texttt{mdot-1} and \texttt{vinf-1}, which have similar wind momentum ratios but the cooling parameters for the WC star differ by a factor of 32.
These simulations are compared to the baseline simulation, which has a radiative post-shock WCR.
%Whilst the simulations with $\eta = 0.01$ still have very imbalanced winds, they are typical of a WR+OB binary with a less intense Wolf-Rayet partner star wind.
All simulations were run for a minimum of 1 orbit.
As these orbits are circular, there should be no major variance of the winds after the start-up transients are fully advected, save for some fluctuations.

\begin{table*}
  \centering
  \begin{tabular}{llllllll}
  \hline
  Name & $\dot{\text{M}}_\text{WR}$ & $\dot{\text{M}}_\text{OB}$ & $v^\infty_\text{WR}$ & $v^\infty_\text{OB}$ & $\eta$ & $\chi_\text{WR}$ & $\chi_\text{OB}$ \\ 
%  &  & \si{\solarmass\per\year} & \si{\centi\metre\per\second} & \si{\centi\metre\per\second} & & & \\ \hline
  \hline
  \texttt{baseline}& \SI{5.0e-6}{\solarmass\per\year} & \SI{5.0e-8}{\solarmass\per\year} & \SI{1e8}{cm.s^{-1}} & \SI{2e8}{cm.s^{-1}} & 0.02 & 1.20 & 1915 \\
  \texttt{mdot-1}  & \SI{1.0e-5}{\solarmass\per\year} & \SI{5.0e-8}{\solarmass\per\year} & \SI{1e8}{cm.s^{-1}} & \SI{2e8}{cm.s^{-1}} & 0.01 & 0.60 & 1915 \\
  \texttt{mdot-2}  & \SI{2.5e-6}{\solarmass\per\year} & \SI{5.0e-8}{\solarmass\per\year} & \SI{1e8}{cm.s^{-1}} & \SI{2e8}{cm.s^{-1}} & 0.04 & 2.39 & 1915 \\
  \texttt{mdot-3}  & \SI{5.0e-6}{\solarmass\per\year} & \SI{1.0e-7}{\solarmass\per\year} & \SI{1e8}{cm.s^{-1}} & \SI{2e8}{cm.s^{-1}} & 0.04 & 1.20 & 957  \\
  \texttt{mdot-4}  & \SI{5.0e-6}{\solarmass\per\year} & \SI{2.5e-8}{\solarmass\per\year} & \SI{1e8}{cm.s^{-1}} & \SI{2e8}{cm.s^{-1}} & 0.01 & 1.20 & 3830 \\
%  \hline
%  \end{tabular}
%  \caption[Mass loss rate series wind parameters]{Wind parameters for simulations varying the mass loss rate, $\dot M$.}
%  \label{tab:mdot-param}
%\end{table*}
%
%\begin{table*}
%  \centering
%  \begin{tabular}{cccccccc}
%  \hline
%  Name & $\dot M_{WR}$ & $\dot M_{OB}$ & $v^\infty_{WR}$ & $v^\infty_{OB}$ & $\eta$ & $\chi_{WR}$ & $\chi_{OB}$ \\ 
%  & \si{\solarmass\per\year} & \si{\solarmass\per\year} & \si{\centi\metre\per\second} & \si{\centi\metre\per\second} & & & \\ \hline
%  \texttt{baseline} & \num{5e-6} & \num{5e-8} & \num{1e8} & \num{2e8} & 0.02 & 1.20 & 1915  \\
  \texttt{vinf-1}   & \SI{5.0e-6}{\solarmass\per\year} & \SI{5.0e-8}{\solarmass\per\year} & \SI{2e8}{cm.s^{-1}} & \SI{2e8}{cm.s^{-1}} & 0.01 & 19.1 & 1915  \\
  \texttt{vinf-2}   & \SI{5.0e-6}{\solarmass\per\year} & \SI{5.0e-8}{\solarmass\per\year} & \SI{5e7}{cm.s^{-1}} & \SI{2e8}{cm.s^{-1}} & 0.04 & 0.07 & 1915  \\
  \texttt{vinf-3}   & \SI{5.0e-6}{\solarmass\per\year} & \SI{5.0e-8}{\solarmass\per\year} & \SI{1e8}{cm.s^{-1}} & \SI{4e8}{cm.s^{-1}} & 0.04 & 1.20 & 30638 \\
  \texttt{vinf-4}   & \SI{5.0e-6}{\solarmass\per\year} & \SI{5.0e-8}{\solarmass\per\year} & \SI{1e8}{cm.s^{-1}} & \SI{1e8}{cm.s^{-1}} & 0.01 & 1.20 & 120   \\
  \hline
  \end{tabular}
  \caption[Terminal velocity series wind parameters]{Wind parameters for simulations varying the wind mass loss rate, $\dot{\text{M}}$, and terminal velocity, $v^\infty$.}
  \label{tab:vinf-param}
\end{table*}

\subsection{Separation distance}

A final series of simulations was performed with the wind parameters equivalent to the baseline model, but with differing orbital separations.
The separation was altered by modifying the orbital period, as stellar masses were to be kept realistic.
The separation distance was varied from the baseline model of \SI{4}{\au} up to \SI{64}{\au} (Table \ref{tab:dsep-param}), which has the effect of modifying the cooling parameter, $\chi$, of each simulation without changing the wind momentum ratio; allowing us to further discern which is the dominant parameter influencing dust formation.
For instance, simulation \texttt{dsep-64AU} has a cooling parameter value approaching the fast WR wind model \texttt{vinf-1}, despite having a wind momentum ratio of 0.02.

%//TODO this section needs some analytical work, specifically on the expected values for dt, things like that, I can also get some stuff from the results

Each simulation has a coarse resolution of $320 \times 320 \times 40$ cells, with a varying number of levels.
As the separation distance is doubled, the associated static mesh refinement box is halved and the number of levels is decremented. This manipulation of levels ensures that the number of cells between the stars is kept consistent and reduces memory usage.
The extent for all simulations in this series were doubled over the other series in this paper to approximately $2000 \times 2000 \times 250 \, \si{\au}$.
Similarly to the previous set of simulations, a minimum of 1 orbit was needed for each simulation, however, as the orbital period of each simulation varies, certain simulations were able to run for a significantly longer length of time, with data for multiple orbits being obtained.

\begin{table*}
  \centering
  \begin{tabular}{lllllll}
    \hline
    Name & P & $d_\text{sep}$ & $\chi_\text{WR}$ & $\chi_\text{OB}$ & Levels & Effective Resolution \\
%    & \si{\year} & \si{\au} & & & & Cells \\
	\hline
    \texttt{dsep-4AU}  & \SI{1.80}{\year} & \SI{4}{\au}  & 1.20 & 1915  & 7 & $(20480 \times 20480 \times 2560) \,\text{cells}$ \\
    \texttt{dsep-8AU}  & \SI{5.06}{\year} & \SI{8}{\au}  & 2.39 & 3830  & 6 & $(10240 \times 10240 \times 1280) \,\text{cells}$ \\
    \texttt{dsep-16AU} & \SI{14.3}{\year} & \SI{16}{\au} & 4.79 & 7659  & 5 & $(5120 \times 5120 \times 640) \,\text{cells}$    \\
    \texttt{dsep-32AU} & \SI{40.5}{\year} & \SI{32}{\au} & 9.57 & 15319 & 4 & $(2560 \times 2560 \times 320) \,\text{cells}$    \\
    \texttt{dsep-64AU} & \SI{115}{\year}  & \SI{64}{\au} & 19.1 & 30638 & 3 & $(1280 \times 1280 \times 160) \,\text{cells}$    \\ \hline
  \end{tabular}
  \caption{Parameters of simulations varying separation distance.}
  \label{tab:dsep-param}
\end{table*}

\subsection{Data collection}

HDF5 files were generated at regular time intervals - 3D HDF5 meshes were generated every $1/100$\ts{th} of an orbit, while 2D slices were produced every $1/1000$\ts{th} of an orbit.
These HDF5 files contain the primitive variables of the simulation: gas density, $\rho$, gas pressure, $P$ and wind velocity components, $v_x$, $v_y$ and $v_z$.
These variables were then used to derive other variables such as temperature and energy.
The scalars governing the dust properties were also stored for each cell: the dust-to-gas mass ratio, $z$, and the dust grain radius, $a$.
The wind ``colour'', the proportion of gas resultant from each star, was also tracked.
A value of 1.0 indicates a pure WR wind while 0.0 indicates a pure OB wind.

The volume-weighted totals of all system parameters were also collected, such as the total gas and dust mass of the system and average grain radius.
 To derive average values, such as $\bar{z}$ and $\bar{a}$, from this data, the values for each are divided by the total system mass.
To calculate dust formation within the WCR, a method of determining if a cell was part of the wind collision region was devised - the cell density would be compared to the predicted density of a single smooth wind with the wind parameters of the WC star in the system:

\begin{equation}
  \rho_\text{WC} = \frac{\dot{M}_{WC}}{4 \pi r^2 v^\infty_{WC}},
\end{equation}

\noindent
where $r$ is the distance from the barycentre.
This threshold value was set to $1.25\rho_\text{SW}$.
Higher threshold values were found to be inaccurate at large distances from the barycentre and other methods of detecting the WCR such as determining wind mixing levels, were not successful in general.

%\begin{figure*}
%  \centering
%  \includegraphics[width=4in]{assets/overdensity-method.png}
%  \caption[Comparison of threshold values for over-density method]{Comparison of threshold values for the over-density method of determining if a cell resides in the wind collision region.
%  A threshold value of $1.25\rho_\text{SW}$ was chosen as it most accurately determined if the cell was in the post-shock region.}
%  \label{fig:overdensity-threshold}
%\end{figure*}

\section{Results}
\label{sec:p1-results}

The first set of simulations were performed in order to assess whether the implemented cooling model would influence dust formation within the WCR.
This was found to be the case, figure \ref{fig:coolingprocess-dustproduction} shows that without cooling only a marginal amount of dust formation occurs.
Dust production for both radiative simulations with cooling processes were significantly higher, with the \texttt{fullcool} simulation having consistently higher dust formation rates than the \texttt{plasmacool} simulation.
This is a sensible result, as figure \ref{fig:postshockcoolcomparison} shows that at immediate post shock temperatures nascent dust grains present can influence immediate post-shock cooling, allowing the wind to reach temperatures low enough for dust formation faster than if only plasma cooling was simulated.

\begin{figure}
  \centering
  \includegraphics[width=\linewidth]{assets/cool-results/cool-phase-dust_rate.pdf}
  \caption[Comparison of dust formation rates with cooling methods]{A comparison of dust formation rates as cooling mechanisms are changed. Without adequate cooling barely any dust is formed, while dust formation does increase with all cooling mechanisms enabled plasma cooling is still the dominant cooling process for dust production.}
  \label{fig:coolingprocess-dustproduction}
\end{figure}

\begin{figure}
  \centering
  \includegraphics[width=\linewidth]{assets/dust-plasma-cooling-comparison/cooling-comparison-forpaper2.pdf}
  \caption[Comparison of dust and plasma cooling rates in post-shock environment]{Comparison of energy loss due to plasma and dust cooling with varying grain sizes in a typical post-shock flow, where $\rho_g = 10^{-16} \, \si{\gram\per\centi\metre\cubed}$ and a dust-to-gas mass ratio of $10^{-4}$ is assumed. Whilst less influential at lower temperatures, dust cooling can aid cooling in the immediate post-shock environment.}
  \label{fig:postshockcoolcomparison}
\end{figure}

% 
In the case of the \texttt{fullcool} simulation, a peak dust formation rate of $\SI{7e-09}{\solarmass\per\year}$ was calculated, this fluctuation appears to be due to dust forming mostly in high density instabilities (figures \ref{fig:coolingprocess-dustproduction} \& \ref{fig:coolingprocess-density}).

\begin{figure}
  \centering
  \includegraphics[width=\linewidth]{assets/results/radiative/radiative-rho.pdf}
  \caption[Instabilities due to cooling]{Density comparison for \texttt{nocool} and \texttt{fullcool} models, with cooling enabled instabilities are far more prevalent, with pockets of very high density material within the WCR.}
  \label{fig:coolingprocess-density}
\end{figure}

% //TODO plot showing simulation a t 1.0, 2.0 and 3.0 phase, showing difference in dust formation

% Temperature and dust formation on leading edge

As cooling is significant in the post-shock WR wind, further compression occurs, resulting in much higher post-shock densities (figure \ref{fig:postshockcompression}).
Gas rapidly cools within this post-shock region, corresponding to where energy losses were stated to be important in \cite{usov_stellar_1991}; this rapid cooling results in ideal conditions for dust formation, especially within high density instabilities.
A similar effect for the OB wind is not observed, as radiative energy losses are not influential on the dynamics of the flow, due to the faster, significantly thinner stellar wind.
Figure \ref{fig:postshocktemperature} shows that the \texttt{fullcool} simulation has a similar immediate-post shock temperature to an adiabatic model, however this region cools within an extremely short timescale, allowing the nascent dust grains to grow.
Figure \ref{fig:full-radiative-z} shows dust clumps forming shortly after initial wind collision, these clumps rapidly convert post-shock gas to dust; however, rapid dust production tapers off as the post-shock flow becomes more diffuse.
This behaviour is similar to previous models, which suggest that the bulk of dust formation occurs only a short distance from the parent stars.
Temperature is also significantly more affected in the leading edge relative to the orbital motion, leading to a larger portion of dust forming in this region.

% It is believed that this effect is due to the region not being within line-of-sight of the WC star, and as such not being subject to the same degree of oblique shocks from the stellar wind.
% Actually it may be an idea to discuss this with Julian, since it could be a number of factors, maybe even orbital motion, or the wind colliding with late-WCR, leading to more compression better cooling? would explain instabilities
% Julian noted in his 2009 paper that a similar principle was observed
% Most likely due to oblique collision, between 
\cite{pittard_3d_2009} notes that in the case of colliding winds with $\eta = 1$ the trailing edge of the WCR takes part in oblique shocks with the stellar winds, while the leading edge is shadowed by the upstream WCR from the colliding material.
This results in a trailing edge with strong instabilities and cool, high density clumps of post-shock wind, while the leading edge has a low density flow that is not dominated by instabilities.
% significant distance preventing oblique shocks
This does not appear to occur in these low-$\eta$ systems, as oblique shocks occur at a much greater distance, when the stellar wind is significantly less dense.
% Need to discuss this a bit with julian, see if my reasoning is sound
Instead, the leading edge of the WCR appears to be much thinner and denser than the trailing edge, this is believed to be due to the leading edge interacting with the outflowing material due to the systems orbital motion, sweeping up material and obliquely shocking with the downstream WCR. % This needs further work, but I think thats a working theory
% Dust formation occurs some distance from the immediate post-shock region, time to cool, in agreement with observations, see williams 1990
% Formation appears to stagnate after this formation period, insufficient density due to diffusion?
Most dust formation occurs in this downstream post-shock region, as soon as it has sufficiently cooled.
Furthermore, dust formation slows significantly as the post-shock wind begins to diffuse, limiting overall dust formation to a region around 100 AU from the WCR apex. % This needs to be more stringent, no time for that right now
This is in agreement with the research by \cite{williams_dust_1990} and \cite{hendrix_pinwheels_2016}, which observed that there is a limited region suitable for dust formation.
% Section needs a diagram showing the described contact surfaces, might be a good idea to name them, alpha beta contact surfaces? Use the eta graphs for baseline

\begin{figure*}
  \centering
  \includegraphics{assets/results/radiative/radiative-crop-2-rho.pdf}
  \caption[Density comparison of simulations with differing radiative processes]{Density comparison of simulations with differing radiative processes.}
  \label{fig:postshockcompression}
\end{figure*}

\begin{figure*}
  \centering
  \includegraphics{assets/results/radiative/radiative-crop-2-temp.pdf}
  \caption[Temperature comparison of simulations with differing radiative processes]{Temperature comparison of simulations with differing radiative processes.}
  \label{fig:postshocktemperature}
\end{figure*}

\begin{figure}
  \centering
  \includegraphics[width=\linewidth]{assets/results/radiative/z.pdf}
  \caption[\texttt{Baseline} simulation $z$, full extent]{Full extent of \texttt{baseline} simulation, showing dust-to-gas mass ratio. Dust typically formed in clumps within instabilities, leading to variation of dust formation as the simulation progresses. Most of the dust forms in the leading arm of the WCR.}
  \label{fig:full-radiative-z}
\end{figure}


% Orbital variation in general, simulation run out to a longer distance, no major variance between orbits

% \begin{figure}
%   \centering
%   \includegraphics{assets/cool-results/cool-phase-avg_a.pdf}
%   \caption[Comparison of grain radii with varying cooling methods]{Average grain radius throughout simulations with varying cooling methods, plasma and dust are close to being the same, however without any cooling processes dust destruction is dominant, resulting in grain radius falling below the initial grain radius, $a_i$.}
%   \label{fig:coolingprocess-grainradius}
% \end{figure}

\begin{table}
  \centering
  \begin{tabular}{llllll}
  \hline
  Model & $\eta$ & $\chi_\text{WR}$ & $\chi_\text{OB}$ & $\dot{\text{M}}_\text{D,avg}$ & $\dot{\text{M}}_\text{D,max}$ \\
   &  &  &  & \si{\solarmass\per\year} & \si{\solarmass\per\year} \\ \hline
  \texttt{baseline} & 0.02   & 1.20 & 1915 & \num{5.38e-10} & \num{9.06E-07} \\ \hline
  \texttt{plasma}   & 0.02   & 1.20 & 1915 & \num{1.29e-10} & \num{9.06E-07} \\
  \texttt{nocool}   & 0.02   & 1.20 & 1915 & \num{2.71e-14} & \num{9.06E-07} \\ \hline
  \end{tabular}
  \caption{Rates of dust production for radiative simulation set.}
  \label{tab:radiative-average-rates}
\end{table}

\subsection{Mass loss rate variation}

% High mass loss rate for both stars results in a high rate of dust formation 
Dust formation in the mass loss variation simulations was found to be dependent on strong winds from either the WC or OB star.
As can be seen in figure \ref{fig:mdotdustproductionrate}, the rates are stratified into similar dust production rates for simulations with increases or decreases in mass loss rates; simulations with either wind being stronger produced the most dust, while simulations with weaker winds produced approximately 3 orders of magnitude less dust than the most productive simulations.
% Note this is not related to the amount of wind, appears to be proportional to the wind momentum ratio instead 
However, the heightened dust production rate does not correspond to the total mass loss rate of the system.
For instance, \texttt{mdot-1} and \texttt{mdot-3} produce on average 2 orders of magnitude more dust with a combined mass loss rate with 1.99 and 1.01 times more wind than the baseline simulation.
% Indicates that cooling is not the only factor, a strong shock must form first
All simulations in this system have low values for $\chi_{WR}$ compared to other simulation sets, implying that cooling is not the only governing factor, and that a strong shock must also form.

\begin{figure}
  \centering
  \includegraphics[width=\linewidth]{assets/mdot-results/mass-loss-phase-dust_rate.pdf}
  \caption[Dust production rate for simulations varying mass loss rate]{A comparison of dust production rates for simulations that vary mass loss rate, $\dot M$, simulations with either a strong primary or secondary wind produce similar levels of dust, whilst if either wind is weaker, dust production rate is reduced.}
  \label{fig:mdotdustproductionrate}
\end{figure}

\begin{table}
  \centering
  \begin{tabular}{llllll}
  \hline
  Model & $\eta$ & $\chi_\text{WR}$ & $\chi_\text{OB}$ & $\dot{\text{M}}_\text{D,avg}$ & $\dot{\text{M}}_\text{D,max}$ \\
   &  &  &  & \si{\solarmass\per\year} & \si{\solarmass\per\year} \\ \hline
  \texttt{baseline} & 0.02   & 1.20 & 1915 & \num{5.38e-10} & \num{9.06E-07} \\ \hline
  \texttt{mdot-1}   & 0.01   & 0.60 & 1915 & \num{8.79E-09} & \num{1.42E-06} \\
  \texttt{mdot-2}   & 0.04   & 2.39 & 1915 & \num{2.53E-11} & \num{5.83E-07} \\
  \texttt{mdot-3}   & 0.04   & 1.20 & 957  & \num{2.34E-08} & \num{1.17E-06} \\
  \texttt{mdot-4}   & 0.01   & 1.20 & 3830 & \num{3.81E-11} & \num{7.11E-07} \\ \hline
  \end{tabular}
  \caption{Rates of dust production for mass loss rate simulation set.}
  \label{tab:mdot-average-rates}
\end{table}

\subsection{Terminal velocity variation}
\label{sec:paper1vinfresults}

\begin{figure}
  \centering
  \includegraphics[width=\linewidth]{assets/vinf-results/vinf-phase-dust_rate.pdf}
  \caption[Comparison of dust production rate for simulations varying wind terminal velocity]{Comparison of dust production rate for simulations varying wind terminal velocity, $v^\infty$. Simulations with strong wind velocity imbalance produce significantly more dust than their counterparts.}
  \label{fig:vinfdustproduction}
\end{figure}

Varying the wind terminal velocity appears to have an extremely strong effect on dust formation, with effects that are not solely related to cooling.
% Slow WR wind = strongly cooling high dust formation
The dust production rate is exceptionally high in the case of \texttt{vinf-2}, which has an extremely slow wind velocity of \SI{500}{\kilo\metre\per\second}, closer to that of a typical LBV star rather than that of a WC.
This very slow, dense wind is highly influenced by radiative cooling in the post-shock environment, driving thermal instabilities and leading to high density pockets of cooled gas.
% Observation shows extremely instability influenced winds, discuss instabilities in general, Kelvin Helmholtz?
This can be seen in figure \ref{fig:vinfrhodcomp}, where \texttt{vinf-2} produces large quantities of dust near apex of the WCR on the WCR side, which is then mixed throughout the WCR.
This flow is highly radiative and is quickly cooled back to the initial wind temperature.
The factor of 4 wind velocity imbalance between the primary and secondary wind creates a strong velocity shear, leading to the formation of Kelvin-Helmholtz instabilities.

% Note increased dust in non-shocked wind, this was found to be a fraction of the overall dust produced
It should be noted that dust production in general increased outside of the WCR in the case of \texttt{vinf-2}, this is largely due to significantly higher wind density within the WC wind, and increased formation time before wind collision.
This dust would be destroyed via the photodissociation process, which is not included in this model, but would be included in future models if this avenue of research is continued.
Despite this, the dust production outside of the WCR does not significantly impact the total dust production rate, and numerical analysis of dust production such as in figure \ref{fig:dsepdustproduction} does not include dust produced outside of the wind collision region.
%//TODO find actual value
In the case of a fast WC wind, dust production effectively ceases,  with an average dust production rate of $\SI{9e-14}{\solarmass\per\year}$, 2 orders of magnitude less than \texttt{vinf-4}, despite having a similar wind momentum ratio.

% Fast OB wind = very strong shock

% \begin{figure}
%   \centering
%   \includegraphics{assets/results/vinf/vinf-finished-rho.pdf}
%   \caption[]{Comparison of terminal velocity variation simulations, simulations with a high ratio of wind velocities appear to exhibit instabilities leading to strong wind mixing, as these seem to be based on velocity shear, it is reasonable to assume that the structure of the WCR is dominated by Kelvin-Helmholtz instabilities.}
%   \label{fig:vinfdensitycomp}
% \end{figure}

\begin{figure}
  \centering
  \includegraphics[width=\linewidth]{assets/results/vinf/vinf-finished-rhod.pdf}
  \caption[Dust density comparison of terminal velocity varying systems]{Comparison of dust density in $v^\infty$ variation simulations, simulations with either a high OB wind velocity or low WC wind velocity produce large quantities of dust. Simulation \texttt{vinf-1}, which has a high velocity WC wind, does not produce any appreciable dust within the WCR. \texttt{vinf-1} and \texttt{vinf-4} have a smoother WCR with less instabilities as both winds have identical terminal speeds, resulting in no velocity shear.}
  \label{fig:vinfrhodcomp}
\end{figure}

% Focus on vinf-3 and vinf-4 figure analysis, zoom out, VINF3 has a very stable secondary wind but much stronger instabilities, implies uncertainties due to velocity differential as well? More dust produced even though mean temperature of this simulation is much higher, implies cooling not the only mechanism, strong shock + additional isntabilities
Simulations \texttt{vinf-3} and \texttt{vinf-4} show that when the secondary wind velocity was altered, drastic changes to the dust formation rate occurred, similar to modifying the mass loss rate of the secondary star.
% Instabilities in secondary wind can 
Instabilities due to the secondary wind appear to be the result of this, a greater secondary wind velocity would lead to a greater velocity shear along the discontinuity, resulting in Kelvin-Helmholtz instabilities in \texttt{vinf-3} but not in \texttt{vinf-4} \citep{stevens_colliding_1992}.
Both \texttt{vinf-2} and \texttt{vinf-3} exhibit KH instabilities, and both have a terminal velocity ratio, $v_\text{OB}^\infty / v_\text{WR}^\infty = 4$.
This would augment the already present thermal instabilities due to radiative cooling, leading to a less ordered, clumpy post-shock environment.
This is found to be the case with \texttt{vinf-3}, which has a far greater amount of dust formation within the WCR, and has a significantly more mixed wind.
In figure \ref{fig:obvinfzcomp} where \texttt{vinf-3} and \texttt{vinf-4} are directly compared, the presence of a much faster secondary wind results in a velocity shear that produces a much broader WCR, with high density pockets formed within instabilities, which appear to produce the bulk of dust, despite both simulations having an adiabatic second wind. 
This suggests that prolific dust formation occurs in a post-shock primary wind shaped by instabilities, produced either from strong radiative cooling, or through a strong velocity shear, leading to K-H instabilities.
Radiative cooling is also important beyond thermal instabilities, reducing temperatures in the high-density immediate post shock flow so that dust can begin to form.
% Stratification of simulations does not occur in the same way, 
Results appear to be stratified somewhat in terms of $\eta$, where simulations where $\eta = 0.04$ produce significantly more dust than simulations with more imbalanced winds (figure \ref{fig:vinfdustproduction}).
However, this dependence is different to the mass loss rate simulation subset, and the stratification is less apparent.
% Direct eta correlation, however fast drop of dust formation for Fast WR wind model


% //TODO add LBV to mass loss rate chart


% Starting to form concept that chi is also not fundamental, cooling not everything, formation of strong instabilities however by any mean is most important


% Purpose of radiative line driving, close binaries may have lower OB wind velocity, leading to less dust formation
% Refer to radiative inhibition 
% Cannot be simulated with current version of code, but interesting to consider

\begin{figure}
  \centering
  \includegraphics[width=\linewidth]{assets/results/vinf/vinf-rhod.pdf}
  \caption[OB terminal velocity wind dust comparison]{Comparison of dust density in simulations with modified OB wind terminal velocities. The simulations are fully advected with $\phi = 3.0$, dust formation and instabilities are far more pronounced in \texttt{vinf-3}, which has an OB wind velocity a factor of 4 larger than \texttt{vinf-4}.}
  \label{fig:obvinfzcomp}
\end{figure}

\begin{figure}
  \centering
  \includegraphics[width=\linewidth]{assets/results/mixed/eta-004-comparison-r0.pdf}
  \caption[Wind colour comparison of $\eta = 0.04$ winds]{Comparison of wind colour in simulations \texttt{vinf-3} and \texttt{mdot-3}. The WR wind has a colour of 1.0 while the OB wind has a colour of 0.0. wind mixing is significantly more pronounced, with a pronounced post-shock WR wind that appears to be strongly influenced by Kelvin-Helmholtz instabilities, due to the increased wind velocity imbalance and lower degree of cooling.}
  \label{fig:eta004comparisoncolour}
\end{figure}

\begin{figure}
  \centering
  \includegraphics[width=\linewidth]{assets/results/mixed/eta-004-comparison-rhod.pdf}
  \caption[]{Comparison of dust density in simulations with a strong secondary wind, \texttt{vinf-3} and \texttt{mdot-3}. Dust in \texttt{vinf-3} is produced to a much higher degree in the trailing edge of the wind, and increased mixing of the wind due to Kelvin-Helmholtz instabilities has led to dust forming throughout the WCR, rather than near the apex of the WCR.}
  \label{fig:eta004comparisonrhod}
\end{figure}

By directly comparing two prolific dust producing models with $\eta = 0.04$, \texttt{vinf-3} and \texttt{mdot-3}, we can see that both WCRs are dominated by instabilities.
\texttt{vinf-3} in particular is more thoroughly mixed (figure \ref{fig:eta004comparisoncolour}).
Iin particular, it has a much larger trailing edge that produces large quantities of dust (figures \ref{fig:eta004comparisonrhod}).
% This additional degree of instability appears to be due to Kelvin-Helmholtz instabilities, which grow significantly as the post-shock gas flows away from the stagnation point, as described in \textcite{stevens_colliding_1992}.
These simulations produce approximately the same amount of dust, with \texttt{vinf-3} also consistently producing dust in the trailing edge of the WCR.
From these results it is clear that the dust production rate is increased if there is a highly imbalanced wind velocity, with a slow WC and fast OB wind, as this leads to a post-shock environment governed by thin-shell and Kelvin-Helmholtz instabilities.

% \begin{figure}
%   \centering
%   \includegraphics{assets/vinf-results/vinf-phase-avg_a.pdf}
% \end{figure}

% //TODO include table of average/peak dust formation rates for simulations with vinf and mdot variation

\begin{table}
  \centering
  \begin{tabular}{llllll}
  \hline
  Model & $\eta$ & $\chi_\text{WR}$ & $\chi_\text{OB}$ & $\dot{\text{M}}_\text{D,avg}$ & $\dot{\text{M}}_\text{D,max}$ \\
   &  &  &  & \si{\solarmass\per\year} & \si{\solarmass\per\year} \\ \hline
  \texttt{Baseline} & 0.02   & 1.20 & 1915  & \num{5.38e-10} & \num{9.06E-07} \\ \hline
  \texttt{vinf-1}   & 0.01   & 19.1 & 1915  & \num{8.88e-13} & \num{7.11E-07} \\
  \texttt{vinf-2}   & 0.04   & 0.07 & 1915  & \num{1.17e-7}  & \num{1.17E-06} \\
  \texttt{vinf-3}   & 0.04   & 1.20 & 30638 & \num{6.30e-11} & \num{1.17E-06} \\
  \texttt{vinf-4}   & 0.01   & 1.20 & 120   & \num{1.94e-8}  & \num{7.11E-07} \\ \hline
  \end{tabular}
  \caption{Rates of dust production for terminal velocity simulation set.}
  \label{tab:vinf-average-rates}
\end{table}

\subsection{Separation variation}

% Direct chi correlation, could also be due to weaker shocks

There is a clear correlation between separation distance and dust formation rate, with dust production drastically increasing as orbital separation is decreased (figure \ref{fig:dsepdustproduction}).
This influence on the dust formation rate is non-linear, with a doubling of the separation distance increasing the dust production rate by approximately one order of magnitude.
Variation of the dust production rate also appears to increase as separation distance is reduced, leading to instances where a simulation may temporarily produce more dust than a simulation with a tighter orbit, such as the case with \texttt{dsep-4AU} and \texttt{dsep-8AU} at $\phi = 0.6$ to $\phi = 0.65$.
As we have previously discussed, instabilities drive slightly intermittent, but highly efficient dust formation, which would explain these fluctuations (figure \ref{fig:dsepinstabilities}).

\begin{figure}
  \centering
  \includegraphics[width=\linewidth]{assets/dsep-results/dsep-phase-dust_rate.pdf}
  \caption[Dust formation rate versus binary separation distance]{A comparison of dust formation rates versus orbital phase for a set of simulations that vary separation distance, $d_\text{sep}$. A clear inverse relationship between separation distance and dust production rate exists, most likely due to the stellar winds becoming more diffuse further from their origin stars, leading to weaker shocks and a WCR that behaves more adiabatically.}
  \label{fig:dsepdustproduction}
\end{figure}

\begin{figure*}
  \centering
  \includegraphics{assets/adiabatic-flow/instab-comp-rho.pdf}
  \caption[A comparison of the structures of simulations varying $d_\text{sep}$]{A comparison of the structures of simulations varying $d_\text{sep}$, the scale of each plot has been changed to allow for a similar feature size, as can be seen simulations with a closer separation distance have collision regions whose structure is more strongly influenced by instabilities, particularly thin-shell instabilities brought on by radiative behaviour within the WCR.}
  \label{fig:dsepinstabilities}
\end{figure*}

% Dust yields
This matches observations of episodic dust forming systems, where infrared emission due to dust is maximised at or shortly after periastron passage. This also lends further evidence that dust formation rates are not influenced solely by the momentum ratio, as this is kept constant, and instead is strongly influenced by the wind density at collision and post-shock cooling. 

\begin{table}
  \centering
  \begin{tabular}{llllll}
  \hline
  Model & $\eta$ & $\chi_\text{WR}$ & $\chi_\text{OB}$ & $\dot{\text{M}}_\text{D,avg}$ & $\dot{\text{M}}_\text{D,max}$ \\
   &  &  &  & \si{\solarmass\per\year} & \si{\solarmass\per\year} \\ \hline
  \texttt{dsep-4AU}  & 0.02   & 1.20 & 1915  & \num{5.38e-10} & \num{9.06E-07} \\ 
  \texttt{dsep-8AU}  & 0.01   & 2.39 & 3830  & \num{4.39e-11} & \num{9.06E-07} \\
  \texttt{dsep-16AU} & 0.04   & 4.79 & 7659  & \num{1.77e-12} & \num{9.06E-07} \\
  \texttt{dsep-32AU} & 0.04   & 9.57 & 15319 & \num{8.83e-14} & \num{9.06E-07} \\
  \texttt{dsep-64AU} & 0.01   & 19.1 & 30638 & \num{2.41e-14} & \num{9.06E-07} \\ \hline
  \end{tabular}
  \caption{Rates of dust production for separation distance set.}
  \label{tab:radiative-average-rates}
\end{table}


\section{Conclusions}
\label{sec:p1-conclusion}

% Dust formation appears to be extremely sensitive to initial conditions, particularly terminal velocity
The simulations in this chapter were conducted over a fairly limited parameter space for mass loss rate and wind terminal velocity.
Despite this, dust production varied by up to 6 orders of magnitude.
Dust formation was found to be extremely sensitive to the wind properties of both stars, which imposes a limited range of wind parameters for dust to form efficiently.
This would explain why these dust forming systems are comparatively rare, compared to the total number of systems with binary massive stars and interacting winds, and also why periodic dust forming systems have eccentric orbits.
% Dust formation rate also varies in accordance with the separation distance, but is somewhat less sensitive to change.
% This would present itself periodic WCd systems, as WC stars do not undergo significant variability of their wind properties over the systems orbital timescale.
% Highly speculative, bring this one down to earth
% This would present an interesting factor in the case of CWB systems with an LBV partner such as $\eta$ Carinae, as an LBV would result in a highly variable wind collision region outside of fluctuations due to separation distance \parencite{nazeChangingWindCollision2018}.
% Briefly discuss extrapolation of results, other systems
The baseline system, WR98a, has a significantly lower mass loss rate than other well-characterised WCd systems, such as WR140 and WR104.
The WC star in WR 140 has a mass loss rate an order of magnitude larger than WR98a, for instance.
% Allude to additional research, covered in papers 2 and 3?
Future research in this topic will cover these systems,
% as an exploration of more varied conditions would be conducive to demonstrating the veracity of this dust model.
to explore how closely they match observations.

\subsection{Wind mixing within the WCR}

While interaction between hydrogen and dust grains is not simulated by the dust model, \cite{leteuffModelDustFormation2002} notes that hydrogen could be a potential catalyst for amorphous carbon grain formation.
Figure \ref{fig:radiative-windmixing} shows that the wind is far more effectively mixed by instabilities if it is sufficiently radiative.
An improved dust model which can calculate grain yields from chemical reactions could be used to investigate this further.
Conveniently, implementation of a chemical model into Athena++ through passive scalars is a future feature in the projects roadmap.
Additionally, a multi-fluid model could be used to model the dynamics of grains, as larger grains are significantly more massive than the surrounding medium, and hence have more inertia, this means they may not be necessarily co-moving in a turbulent wind environment.
% Improvements to model, multi-wind model, astrochemical model? Extremely large bounds for future work
% Mention single wind WC grain formation

\begin{figure}
  \centering
  \includegraphics[width=\linewidth]{assets/results/radiative/radiative-r0.pdf}
  \caption[Wind mixing due to radiative methods]{Wind ``colour'' for \texttt{nocool} and \texttt{fullcool} models. The WCR is more thoroughly mixed if the simulation is allowed to cool.}
  \label{fig:radiative-windmixing}
\end{figure}

\section{Summary}

A parameter space exploration of Colliding Wind Binary systems undergoing dust formation has proven to yield fascinating insights into how dust forms within the WCR.
Dust production within these systems is poorly understood, and with direct observations of the WCR rendered difficult by the extreme conditions of these systems, it falls on numerical simulation to elucidate the post-shock conditions.
Most interesting of all is how sensitive to changing wind conditions this dust production is.
This parameter space exploration, whilst quite conservative, resulted in a change in dust formation rates of up to 6 orders of magnitude.
In all simulations, the bulk of dust formation was found to occur within high-density pockets formed through thin-shell or Kelvin-Helmholtz instabilities, suggesting that strong cooling and a fast secondary wind are both important factors for dust production.
For high levels of dust formation, an ideal system should have a slow, dense primary wind and a fast, dense secondary wind, with a close orbit.
A combination of these properties ensures the formation of dense pockets of cool post-shock gas ideal for dust formation.

There is significant potential for additional research in this field.
Parameter mixing was not performed, due to the simulation time required for producing many more simulations, but performing examples on more extreme systems, such as those with a LBV primary star or a WR+WR system is a potential avenue of research.
Future work could introduce additional dust formation and destruction mechanisms, such as grain-grain collision or photodissociation.
Modelling effects such as radiative line driving or use of a multi-fluid model could also prove fruitful. 
Another interesting avenue of research is the simulation of eccentric, periodic dust forming systems; simulating either an entire or a partial orbit of a system such as WR140 would be a logical next step for this project.

\section{Acknowledgements}

This work was undertaken on ARC4, part of the High Performance Computing facilities at the University of Leeds, UK.
We thank the publication referee for feedback and corrections to the text.

%%%%%%%%%%%%%%%%%%%% REFERENCES %%%%%%%%%%%%%%%%%%


\bibliographystyle{mnras}
\bibliography{references.bib} % if your bibtex file is called example.bib

% Don't change these lines
\bsp	% typesetting comment
\label{lastpage}
\end{document}

% End of mnras_template.tex
