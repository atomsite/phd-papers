% mnras_template.tex 
%
% LaTeX template for creating an MNRAS paper
%
% v3.0 released 14 May 2015
% (version numbers match those of mnras.cls)
%
% Copyright (C) Royal Astronomical Society 2015
% Authors:
% Keith T. Smith (Royal Astronomical Society)

% Change log
%
% v3.0 May 2015
%    Renamed to match the new package name
%    Version number matches mnras.cls
%    A few minor tweaks to wording
% v1.0 September 2013
%    Beta testing only - never publicly released
%    First version: a simple (ish) template for creating an MNRAS paper

%%%%%%%%%%%%%%%%%%%%%%%%%%%%%%%%%%%%%%%%%%%%%%%%%%
% Basic setup. Most papers should leave these options alone.
\documentclass[fleqn,usenatbib]{mnras}

% MNRAS is set in Times font. If you don't have this installed (most LaTeX
% installations will be fine) or prefer the old Computer Modern fonts, comment
% out the following line
% Depending on your LaTeX fonts installation, you might get better results with one of these:
% \usepackage{mathptmx}
% \usepackage{txfonts}

% Use vector fonts, so it zooms properly in on-screen viewing software
% Don't change these lines unless you know what you are doing
\usepackage[T1]{fontenc}

% Allow "Thomas van Noord" and "Simon de Laguarde" and alike to be sorted by "N" and "L" etc. in the bibliography.
% Write the name in the bibliography as "\VAN{Noord}{Van}{van} Noord, Thomas"
\DeclareRobustCommand{\VAN}[3]{#2}
\let\VANthebibliography\thebibliography
\def\thebibliography{\DeclareRobustCommand{\VAN}[3]{##3}\VANthebibliography}


%%%%% AUTHORS - PLACE YOUR OWN PACKAGES HERE %%%%%

% Only include extra packages if you really need them. Common packages are:
\usepackage{graphicx}	% Including figure files
\usepackage{amsmath}	% Advanced maths commands
\usepackage{amssymb}	% Extra maths symbols
\usepackage{newtxtext,newtxmath}

%%%%%%%%%%%%%%%%%%%%%%%%%%%%%%%%%%%%%%%%%%%%%%%%%%

%%%%% AUTHORS - PLACE YOUR OWN COMMANDS HERE %%%%%

% Packages for slightly better tables
\usepackage{booktabs}
% SI units package
\usepackage{siunitx} % Better units, especially SI
% Units used in work
\DeclareSIUnit[]\solarmass
{\text{\ensuremath{\textup{M}_{\odot}}}}
\DeclareSIUnit[]\solarluminosity
{\text{\ensuremath{\textup{L}_{\odot}}}}
\DeclareSIUnit[]\solarradius
{\text{\ensuremath{\textup{R}_{\odot}}}}
\DeclareSIUnit[]\year
{\text{yr}}
\DeclareSIUnit[]\au
{\text{AU}}
\DeclareSIUnit[]\parsec
{\text{pc}}
\DeclareSIUnit[]\erg
{\text{erg}}
\DeclareSIUnit[]\arcsecond
{\text{as}}

% Author commands
\newcommand{\ts}{\textsuperscript}

% Math macros 
\newcommand{\swr}{\ensuremath{_{\text{WR}}}}
\newcommand{\sob}{\ensuremath{_{\text{OB}}}}
\newcommand{\rms}[1]{\ensuremath{_{\text{#1}}}}

% Please keep new commands to a minimum, and use \newcommand not \def to avoid
% overwriting existing commands. Example:
%\newcommand{\pcm}{\,cm$^{-2}$}	% per cm-squared

%%%%%%%%%%%%%%%%%%%%%%%%%%%%%%%%%%%%%%%%%%%%%%%%%%

%%%%%%%%%%%%%%%%%%% TITLE PAGE %%%%%%%%%%%%%%%%%%%

% Title of the paper, and the short title which is used in the headers.
% Keep the title short and informative.
\title[Hydrodynamical Simulation of WR140]{Exploring dust formation in the episodic WCd system WR140}

% The list of authors, and the short list which is used in the headers.
% If you need two or more lines of authors, add an extra line using \newauthor
\author[J. W. Eatson et al.]{
J. W. Eatson\thanks{E-mail: \href{mailto:py13je@leeds.ac.uk}{py13je@leeds.ac.uk}} \&
J. M. Pittard
\\
School of Physics and Astronomy, University of
       Leeds, Woodhouse Lane, Leeds LS2 9JT, UK\\  
}

% These dates will be filled out by the publisher
\date{Accepted XXX. Received YYY; in original form ZZZ}

% Enter the current year, for the copyright statements etc.
\pubyear{2022}

% Don't change these lines
\begin{document}
\label{firstpage}
\pagerange{\pageref{firstpage}--\pageref{lastpage}}
\maketitle

% Abstract of the paper
\begin{abstract}
\noindent


\end{abstract}

% Select between one and six entries from the list of approved keywords.
% Don't make up new ones.
\begin{keywords}
stars: Wolf-Rayet -- methods: numerical -- binaries: general
\end{keywords}

%%%%%%%%%%%%%%%%%%%%%%%%%%%%%%%%%%%%%%%%%%%%%%%%%%

%%%%%%%%%%%%%%%%% BODY OF PAPER %%%%%%%%%%%%%%%%%%

\section{Introduction}

The dynamics of massive stars in binary systems is a particularly fascinating subject.
These incredibly violent systems are obscured behind vast clouds of outflowing stellar wind, the result of the most massive stars we know of slowly tearing themselves asunder.
Despite powerful winds, extreme temperatures, and intense astrophysical shocks from these colliding winds, interstellar dust

% Observational history, define acronyms commonly used


% Dust production yields

In fact, these dust producing CWB (WCd) systems can produce an extreme quantity of dust.
Typically around $1\%$ of the stellar wind is converted into dust a short amount of time after wind collision, but in the case of more prolific systems such as WR104 this can be as high as $36\%$ \citep{lauRevisitingImpactDust2020}.
This corresponds to dust production rates on the order of $10^{-6} \, \si{\solarmass\per\year}$, rivalling other profuse dust producing phenomena such as AGB stars.

% Difficulty of detailed observation
While we know that truly colossal amounts of dust form in these systems through infrared observation, we find that the mechanisms involved in 

% Dust formation particulars
% Dust forms in wcr
Observations conclude that dust forms exclusively within the WCR of the system,
% Dust forms close to system
observations also indicate that dust formation occurs rapidly and close to the system
\citep{williamsInfraredPhotometryLatetype1987,williamsMultifrequencyVariationsWolfrayet1990},

this is further backed up by grain 

% Bringing it all together, theories as to how dust formation occurs, density, shielding etc.

% What we intend to do in this project

In this paper we will discuss our methodology in section 

\section{Methodology}
\label{sec:paper-2-methodology}


The periodic dust forming system WR140 was simulated using a fork of the Athena++ hydrodynamical code \citep{stoneAthenaAdaptiveMesh2020}, a series of modifications were implemented to simulate binary system orbits, stellar wind outflows and dust evolution.
These simulations were conducted in 3D in a Cartesian co-ordinate system.
The code solves a Riemann problem at each cell interface to determine the time-averaged values at the zone interfaces, and then solves the equations of hydrodynamics:

\begin{subequations}
  \begin{align}
    \frac{\partial\rho}{\partial t} & +\nabla \cdot \left(\rho \boldsymbol{u}\right) = 0 , \\
    \frac{\partial \rho \boldsymbol{u}}{\partial t} & + \nabla \cdot \left(\rho \boldsymbol{u} u + P \right) = 0, \\
    \frac{\partial \rho \varepsilon}{\partial t} & + \nabla \cdot \left[ \boldsymbol{u} \left( \rho\varepsilon + P \right) \right] = \dot E_\text{cool} , 
  \end{align}
\end{subequations}

\noindent
where $\varepsilon$ is the total specific energy ($\varepsilon = \boldsymbol{u}^2/2 + e/\rho $), $\rho$ is the gas density, $e$ is the internal energy density, $P$ is the gas pressure and $u$ is the gas velocity.
In order to simulate radiative losses, the parameter $\dot E_\text{cool}$ is included, which is the energy loss rate per unit volume from the fluid due to gas and dust cooling.

Spatial reconstruction using a piecewise linear method was performed, while the time-integration scheme is a third-order accurate, three-stage strong stability preserving Runge-Kutta\footnote{SSPRK (3,3)} method \citep{gottliebHighOrderStrong2009}.
Several passive scalars are utilised to model wind mixing and dust evolution, the scalar values are transported by the fluid, for a given scalar species $i$, the scalar is advected through the scalar through the following equation:

\begin{equation}
  \rho \frac{dC_i}{dt} = \frac{\partial}{\partial t} \left( \rho C_i \right) + \nabla \cdot \left( C_i \rho \mathbf{u} \right) = -\nabla \cdot \mathbf{Q}_i ,  
\end{equation}

\noindent
where $\mathbf{Q}_i = - \nu_{ps} \rho \nabla C_i$ is the diffusive flux density and $\nu$ is the passive scalar diffusion coefficient \citep{stoneAthenaAdaptiveMesh2020}.

% Cover mapping on winds

Stellar winds are simulated by modifying the density, $\rho_R$, momentum, $p_R$, and energy, $E_R$ in a small region around both stars.
Winds flow from this ``remap'' region at the stars wind terminal velocity, $v^\infty$. Remap zone parameters are calculated with the formulae

\begin{subequations}
  \begin{align}
    \rho_R & = \frac{\dot M}{4 \pi r^2 v_\infty} , \\
    % P_R    & = \frac{\rho_R}{\mu m_H} k_B T_w , \\
    p_R    & = \rho_R v_{r} , \\
    E_R    & = \frac{P_R}{\gamma - 1} + \frac{1}{2} \rho_R v_\infty^2 ,
  \end{align}
\end{subequations}

\noindent
where $P_R$ is the cell pressure, $P_R = \rho_R k_\text{B} T_w / \mu m_\text{H}$, $T_w$ is the wind temperature, $\mu$ is the mean molecular mass, $m_\text{H}$ is the mass of a hydrogen atom, $v_R$ is the wind velocity as it flows radially from the center of the ``remap zone'' and $r$ is the distance from the current cell to the centre of the remap zone.
This method produces radially out-flowing winds from the star with an expected density and velocity.
This method is stable against numerical instability, while also allowing us to precisely control the winds.

Line driving and wind acceleration effects are not simulated, which can result in divergence with the correct wind velocity as stars approach periastron passage.
Instead, winds are instantaneously accelerated to their terminal velocity.
Additionally, influence from either gravitational self-interaction and interaction with the stars gravity wells is not simulated, with the stellar winds assumed to be travelling far in excess of the system escape velocity.

Athena++ utilises Message Passing Interface (MPI) parallelism.
The numerical problem is broken into blocks, which are distributed between processing nodes on a High Performance Compute (HPC) cluster.
The block size is variable, but for these simulations a block size of $40\times 40 \times 10$ cells in $XYZ$ was found to be optimal.
Adaptive Mesh Refinement was considered for this simulation, however a known issue with the Athena++ code prevented this from being possible.
Passive scalars incorporated into the simulation were found to not be conserved along the interfaces between mesh blocks undergoing refinement, this meant that the simulation would behave non-physically (This bug is recorded as issue \#365 on the Athena++ Github repository\footnote{\texttt{\href{https://github.com/PrincetonUniversity/athena/issues/365.}{https://github.com/PrincetonUniversity/athena/issues/365}}}).
A ring of refined cells across the orbital path was considered, but the performance improvements of this method were found to be negligible and not worth pursuing, as the block based refinement method of Athena++ would result in much redundant refinement.
Instead, a static mesh is used, where the stars predicted orbit over the simulation is refined to the maximum level, with a gradual de-refinement away from this refinement region.

\subsection{Radiative cooling}

Cooling is simulated via the removal of energy from a cell at each time-step.
A cooling rate, $dE/dt$, is calculated, and integrated using a sub-stepping Euler method, with the number of sub-steps determined by the estimated cooling timescale of the cell.
Cooling due to gas and plasma emission in the stellar winds are calculated via individual lookup tables from each wind.
These lookup tables contain the normalised emissivity, $\Lambda\rms{w}(T)$ for a specific temperature from $10^4$ to $10^9 \, \si{\kelvin}$.
The cooling rate is determined for a cell by calculating the cell temperature, and estimating $\Lambda\rms{w}(T)$ using linear interpolation between the nearest emissivity values in the lookup table.
The energy loss is then calculated through the equation

\begin{equation}
  \frac{dE}{dt} = \left(\frac{\rho}{m\rms{H}}\right)^2 \Lambda\rms{w}(T),
\end{equation}

\noindent
where $\rho$ is the gas density and $m\rms{H}$ is the mass of hydrogen.
The lookup table was generated by mixing a series of cooling curves from MEKAL simulations of elemental gasses.
These curves were combined based on the elemental abundances in the WC and OB winds.
To save calculation time, temperatures between $\SI{1e4}{\kelvin} < T \leq \SI{1.1e4}{\kelvin}$ are set to \SI{1e4}{\kelvin} as they are assumed to be either rapidly cooling or a part of the stellar wind outside of the WCR.

\subsection{Dust model}

In order to simulate dust evolution in WR140, included in the hydrodynamical code is a passive scalar dust model that simulates dust growth through collisions between dust grains and carbon atoms, and destruction through sputtering from a hot wind.
The dust model operates on passive scalars, and as such simulates dust that is co-moving with the stellar wind.
Two scalars are used to describe dust in a cell, $a$, the grain radius in \si{\micro\metre}, and $z$, the grain dust-to-gas mass ratio:

\begin{equation}
  z = \frac{\rho\rms{d}}{\rho\rms{g}},
\end{equation}

\noindent
where $\rho\rms{d}$ is the dust density in a cell and $\rho\rms{g}$ is the gas density in a cell.
A number of assumptions are made in this dust model, for instance, the dust grains in the model are spherical, with a uniform density.
Furthermore, dust grains are assumed to have a single size in a region, as well as a constant number density.
As such, this model does not simulate grain agglomeration and fracturing.
Additional mechanisms for dust formation and destruction could also be implemented such as grain-grain agglomeration and photoevaporation.
A multi-fluid model with drag force coupling could also be implemented, however this is beyond the scope of this paper.

Dust is grown through grain accretion using formulae described by \citep{spitzer_jr._physical_2008} where dust grains grow via low-velocity collisions with surrounding carbon atoms, causing them to accrete onto the surface of the dust grain.
Carbon is removed from the gas, reducing the cell density, while the corresponding dust density increases.
This ensures that mass is preserved in the simulation.
Assuming a single average grain size the rate of change in the grain radius in a cell, $da/dt$, is given by the equation:

\begin{equation}
  \frac{da}{dt} = \frac{\xi \rho\rms{C} w\rms{C}}{4\rho\rms{gr}},
\end{equation}

\noindent
where $\xi$ is the grain sticking factor, $\rho\rms{C}$ is the carbon density ($\rho\rms{C} = X\rms{C} \rho\rms{g}$), $w\rms{C}$ is the Maxwell-Boltzmann RMS velocity for carbon ($w\rms{C} = \sqrt{3k\rms{B} T / 12m\rms{H}}$), $k\rms{B}$ is the Boltzmann constant and $\rho\rms{gr}$ is the grain bulk density.
The rate of change in grain mass due to accretion, $dm\rms{gr,ac}/dt$, is calculated with the formulae:

\begin{equation}
  \frac{d m\rms{gr,ac}}{dt} = 4 \pi \rho\rms{gr} a^2 \frac{da}{dt} = \pi \xi \rho\rms{C} w\rms{C} a^2, \\
\end{equation}

\noindent
A bulk density approximating that of amorphous carbon grains ($\rho\rms{gr} = \SI{3.0}{\gram\per\centi\metre\cubed}$) is used for this simulation.

Dust destruction gas-grain sputtering is calculated using the \cite{drainePhysicsDustGrains1979} prescription.
Within a flow of number density $n\rms{g}$ a dust grain of radius $a$ has a grain lifespan, $\tau\rms{gr}$ of:

\begin{equation}
  \tau\rms{gr} = \frac{a}{\dot{a}} \approx \num{3e6} \frac{a}{n\rms{g}} \, \si{\year} .
\end{equation}

\noindent
This value is based on an average lifetime of carbon grains in an interstellar shock with a temperature of $\SI{1e6}{\kelvin} \leq T \leq \SI{3e8}{\kelvin}$ \citep{tielens_physics_1994,dwekCoolingSputteringInfrared1996}.
The rate of change in the dust grain mass due to sputtering, $dm\rms{gr,sp}/dt$, can then be calculated with a similar formulae to the rate of change in grain mass due to accretion:

\begin{equation}
  \frac{dm\rms{gr,sp}}{dt} = 4\pi \rho\rms{gr} a^2 \frac{da}{dt} = - 4 \pi \tau\rms{gr} n\rms{g} a^2.
\end{equation}

\noindent
Finally, the total rate of change in grain mass is calculated, the overall change in dust density is then calculated:

\begin{equation}
  \frac{d \rho\rms{d}}{dt} = \left( \frac{d m\rms{gr,acc}}{dt} + \frac{d m\rms{gr,sp}}{dt}\right) n\rms{d}, 
\end{equation}

\noindent
where $n\rms{d}$ is the dust grain number density.

Cooling via emission of photons from dust grains is also included in this model.
The rate of cooling is calculated using the uncharged grain case of the prescription described in \citep{dwek_infrared_1981}.
Grains are collisionally excited by collisions with ions and electrons, causing them to radiate.
Similarly to the gas/plasma emission model used, the emitted photons are not re-adsorbed by the WCR medium, causing energy to be removed from the simulation.
This therefore makes the assumption that the WCR is optically thin to far-infrared photons, which is observationally correct.
The grain heating rate (in \si{\erg\per\second}) for a dust grain is calculated with the formulae:

\begin{equation}
  \label{eq:p2-grainheat}
  H = 1.26 \times 10^{-19} \frac{n\rms{g}}{A^{1/2}} a^2(\si{\micro\metre}) T^{3/2} h(a,T) , 
\end{equation}

\noindent
where $H$ is the heating rate due to atom and ion collisions, 
$n$ is the particle number density,
$A$ is the mass of the incident particle in AMU,
$a(\si{\micro\metre})$ is the grain radius in microns,
$T$ is the temperature of the ambient gas,
and $h(a,T)$ is the effective grain heating factor.
Individual heating rates for hydrogen, helium, carbon, nitrogen and oxygen are calculated, in order to calculate the total ion collisional heating, $H\rms{coll}$:

\begin{equation}
  H\rms{coll} = H\rms H + H \rms{He} + H\rms C + H\rms N + H\rms O .
\end{equation}

\noindent
The effective grain heating factor for each element is calculated via the equation:

\begin{equation}
  h(a,T) = 1 - \left( 1 + \frac{E^*}{2 k\rms{B} T} \right) e^{- E^* / k\rms{B} T} ,
\end{equation}

\noindent
where $E^*$ is the critical energy required to overcome the grain's potential (Table \ref{tab:p2-criticalenergy}).
The rate of heating due to electron-grain collisions, $H\rms{el}$, is similar to Eq. \ref{eq:p2-grainheat}.
The grain heating factor for electron collisions, $h\rms{e}$, is calculated via an approximation from \cite{dwek_infrared_1981}.
This approximation is performed as a complex integration for every cell and cooling step would need to be performed instead, which was found to take up $>90\%$ of the processing time for a cell.
$h\rms{e}$ is estimated through the following conditions:

\begin{equation}
  \begin{alignedat}{3}
    h\rms{e}(x^*) & = 1 ,                && ~~ x^* > 4.5, \\
             & = 0.37{x^*}^{0.62} , && ~~ x^* > 1.5 , \\
             & = 0.27{x^*}^{1.50} , && ~~ \text{otherwise,}
  \end{alignedat}
\end{equation}

\noindent
where $x^* = \num{2.71e8} a^{2/3} (\si{\micro\metre})/T$.
This approximation differs from the integration method by less than 8\% while being 4 orders of magnitude faster.
Excitation due to grain-grain collisions were not modelled, due to the limitations of the passive scalar model.
In order to calculate the change in energy due to dust cooling, we find the radiative emissivity for dust, $\Lambda\rms{d}(T,a)$, to be:

\begin{equation}
  \Lambda(T,a) = \frac{H\rms{coll} + H\rms{el}}{n\rms{H}}
\end{equation}

\noindent
where $n\rms{H}$ is the number density of hydrogen in the gas.
The energy loss rate from dust cooling, $dE\rms{d}/dt$, then calculated with the equation:

\begin{equation}
  \frac{dE\rms{d}}{dt} = n\rms{T} n\rms{d} \Lambda\rms{d} (T,a) , 
\end{equation}

\noindent
and summated with the gas/plasma energy loss rate.

\begin{table}
  \centering
  \begin{tabular}{ll}
    \hline
    Particle & $E^*$ \\
    \hline
    $e^-$ & $23 \, a^{2/3}(\si{\micro\metre})$ \\
    H     & $133 \, a(\si{\micro\metre})$ \\
    He    & $222 \, a(\si{\micro\metre})$ \\
    C     & $665 \, a(\si{\micro\metre})$ \\
    N     & $665 \, a(\si{\micro\metre})$ \\
    O     & $665 \, a(\si{\micro\metre})$ \\
    \hline
  \end{tabular}
  \caption[Grain potential critical energy]{Grain potential critical energy, $E^*$, for a dust grain of $a$ in \si{\micro\metre} for electrons, $e^-$, as well as the elements considered for grain cooling. The values for carbon, oxygen and nitrogen are identical.}
  \label{tab:p2-criticalenergy}
\end{table}

\section{WR140 and simulation parameters}
\label{sec:paper2-wr140}

% Discuss importance of WR 140 system

WR 140 was simulated in this paper as it is considered to be the archetypical episodic CWB system, whose infrared dust emission peaks around periastron passage.
WR 140 deviates from WR 98a and WR 104 by having an extremely eccentric orbit, which significantly effects the cooling parameter as the orbit progresses.
Additionally, the minimum value for $\chi$ is significantly larger than the other systems, and hence cooling would be less dominant on the dynamics of the WCR, even at periapsis.

% Discuss Chi and Eta

Though these simulations do not calculate wind acceleration due to radiative line driving, both stellar winds are expected to be accelerated close to their terminal wind velocities \citep{lamersIntroductionStellarWinds1999}.
However, this discrepancy should be noted when considering the results of this paper.

\subsection{System parameters}

% Discuss simulation parameters
Recent improved estimations of the orbital parameters of WR140 by \cite{thomasOrbitStellarMasses2021} were used to calculate the orbital path for these simulations, while the mass loss rate, $\dot{\text{M}}$, and the wind terminal velocity, $v^\infty$, were derived from \cite{williamsMultifrequencyVariationsWolfrayet1990}
(Table \ref{tab:wr140systemparameters}).
% Composition
In order to correctly calculate cooling and dust growth, the abundances of hydrogen, helium, and metals, particularly CNO must be included in the simulations parameters.
A typical wind composition for WC stars was assumed for the Wolf-Rayet star, while a solar abundance was assumed for the OB star (Table \ref{tab:p2-abundances}).
The system orbit was calculated using a Keplerian orbital model with the two stars as point-masses.
% Self-gravity of the winds were not considered, as the winds leave the numerical grid quickly, and are travelling at a terminal velocity far in excess of the wind escape velocity, $v^\infty \gg v\rms{esc}$.

\begin{table}
  \centering
  \begin{tabular}{lll}
    \hline
    Parameter & Value & Citation \\
    \hline
    $\text{M}_\text{WR}$ & \SI{10.31}{\solarmass} & \cite{thomasOrbitStellarMasses2021} \\
    $\text{M}_\text{OB}$ & \SI{29.27}{\solarmass} & \cite{thomasOrbitStellarMasses2021} \\
    $P$ & \SI{7.926}{\year} & \cite{thomasOrbitStellarMasses2021} \\
    $e$ & 0.8993 & \cite{thomasOrbitStellarMasses2021} \\
    $\dot{\text{M}}_\text{WR}$ & \SI{5.6e-5}{\solarmass\per\year} & \cite{williamsMultifrequencyVariationsWolfrayet1990} \\
    $\dot{\text{M}}_\text{WR}$ & \SI{1.6e-6}{\solarmass\per\year} & \cite{williamsMultifrequencyVariationsWolfrayet1990} \\
    $v^\infty_\text{WR}$ & \SI{2.86e3}{\kilo\metre\per\second} & \cite{williamsMultifrequencyVariationsWolfrayet1990} \\
    $v^\infty_\text{OB}$ & \SI{3.20e3}{\kilo\metre\per\second} & \cite{williamsMultifrequencyVariationsWolfrayet1990} \\
    $\eta$ & 0.031 & Calculated \\
    $\chi_\text{min}$ & 2.69 & Calculated \\
    \hline
  \end{tabular}
  \caption[WR140 system parameters]{WR140 system parameters.}
  \label{tab:wr140systemparameters}
\end{table}

\begin{table}
  \centering
  \begin{tabular}{lll}
  \hline
  Element & Solar & WC \\ \hline
  $X\rms H   $ & $0.705$ & $0.000$ \\
  $X\rms{He} $ & $0.275$ & $0.546$ \\
  $X\rms C   $ & $0.003$ & $0.400$ \\
  $X\rms N   $ & $0.001$ & $0.000$ \\
  $X\rms O   $ & $0.010$ & $0.050$ \\
  \hline
  \end{tabular}
  \caption[Abundances by mass used for OB and WR stars]{Abundances used for the OB and WR stars being simulated. Other elements are assumed trace when calculating dust emission \citep{williamsSpectraWC9Stars2015}.}
  \label{tab:p2-abundances}
\end{table}

\subsection{Simulation parameters}

% Discuss more simulation parameters

A domain of $128 \times 128 \times 16 \, \si{\au}$ was used for this simulation, with a coarse simulation resolution of $400\times 400 \times 50$ in the XYZ domain.
This simulation has an XY to Z aspect ratio of 8:1 in order to reduce processing time, as the bulk of dust formation was expected to occur a short distance from the WCR.
Due to computing limitations, a complete orbit could not be completed without AMR, instead, a section of the systems orbit, corresponding to an orbital phase of $0.95 \leq \Phi \leq 1.10$ was simulated, where $\Phi$ is the orbital phase.
This section represents the period prior to periastron passage, as well as a brief period after periastron
(Fig. \ref{fig:p2-trajectory}).
This represents a period of approximately \num{1.2} years of the systems orbit, and the period where much of the dust forms \citep{crowther_dust_2003}.
Fig. \ref{fig:p2-orbitalpath} shows the orbital path overlaid onto the statically refined numerical grid, the area of maximum refinement is around the orbital paths of the stars from $0.94 \leq \Phi \leq 1.11$, in order to ensure that the stars are maximally refined.
If the stars leave the maximally refined region of the simulation unphysical behaviour with regards to wind mapping and dust formation occur, as such the simulation is halted when $\Phi = 1.10$.
The simulation was run with two different numerical integrators, a 3\ts{rd} order accurate Runge-Kutta integrator (\texttt{rk3}) and a 4\ts{th} order accurate, 5-stage, 3 storage register strong stability preserving Runge-Kutta integrator (\texttt{ssprk5\_4})
\citep{ruuthHighOrderStrongStabilityPreservingRungeKutta2005}.
The \texttt{ssprk5\_4} integrator was found to be approximately 60\% slower, but markedly more stable.
Prior to periastron passage the \texttt{rk3} integrator was used for its speed, but increasing numerical instability as the stars grew closer resulted in this proving untenable, and was switched to \texttt{ssprk5\_4}.

Over periastron passage the average time-step was found to reduce by an order of magnitude, resulting in a corresponding increase to simulation time (Fig. \ref{fig:p2-timestep}).
At the most numerically complex portion of the simulation, a Courant number of $C = 0.04$ had to be used instead of the initial value of $C = 0.15$, in order to preserve numerical stability.
As the simulation moved past periastron the Courant number was increased every 24 hours of wall time, until $C$ returned to the initial value.
The simulation was conducted on the ARC4 HPC cluster at the University of Leeds with 128 cores.
The code was compiled using the Intel \texttt{ICPC} compiler using \texttt{AVX512} optimisations and the Intel MPI library.

\begin{figure}
  \centering
  \includegraphics[width=\linewidth]{assets/wr140-path.pdf}
  \caption[Simulation orbital trajectories of WR140 WC7 and O5 stars]{Simulation orbital trajectories of the WC7 and O5 stars in WR140. The solid line represents the orbital phase being simulated, corresponding to $0.95 \leq \Phi \leq 1.10$. The simulation starting position for each star has been annotated.}
  \label{fig:p2-trajectory}
\end{figure}

\begin{figure}
  \centering
  \includegraphics[width=\linewidth]{assets/wr140-grid/grid-orbit.pdf}
  \caption{Numerical grid of the WR140 system simulation, static mesh refinement was used to increase the resolution around the orbital path from $0.95 \leq \Phi \leq 1.10$. The orbital path of both stars are overlaid onto this numerical grid. Beyond $\Phi = 1.11$ the WR star exits the fully refined region of the simulation, causing the simulation to break down.}
  \label{fig:p2-orbitalpath}
\end{figure}

\begin{figure}
  \centering
  \includegraphics[width=\linewidth]{assets/wr140-dt.pdf}
  \caption[Timestep over WR140 simulation]{Average timestep, $dt$, over the course of the WR140 simulation, binned every $\Phi = 0.001$. after periastron passage the \texttt{ssprk5\_4} numerical integrator and a drastically reduced Courant number was adopted in order to preserve numerical stability. This increased simulation time by approximately an order of magnitude.}
  \label{fig:p2-timestep}
\end{figure}

\subsection{Data collection}
% Data collection 
Simulation data was exported in the form of HDF5 at regular time intervals - 3D HDF5 meshes were collected every increment of $\Phi = \num{1.5e-3}$, while 2D slices in the XY plane were collected every increment of $\Phi = \num{1.5e-4}$.
These HDF5 files contain the primitive variables of the simulation: gas density, $\rho$, gas pressure, $P$, and wind velocity components, $v_x$, $v_y$ and $v_z$.
These variables were then used to derive other variables such as temperature and energy.
The scalars governing the dust properties were also stored for each cell: the dust-to-gas mass ratio, $z$, and the dust grain radius, $a$.
The wind ``colour'', the proportion of gas from each star, was also tracked.
A value of 1.0 indicates a pure WR wind while 0.0 indicates a pure OB wind.
The volume-weighted totals of all parameters of interest were also collected, such as the average values for $z$, $a$ and the dust production rate within the WCR, $\dot{\text{M}}\rms{d}$.
To calculate $\dot{\text{M}}\rms{d}$, a cell must be identified as being within the WCR, this was performed by comparing the cell density to the predicted density of a single wind with the wind parameters of the WC star in the system.
Any cell with a density higher a certain threshold value was flagged as being within the WCR.
the single-wind density was calculated using the equation:

\begin{equation}
  \rho\rms{SW} = \frac{\dot{\text{M}}\rms{SW}}{4\pi r^2 v^\infty\rms{SW}},
\end{equation}

\noindent
where $r$ is the distance from the barycentre.
This threshold value was set to $\rho\rms{thres} = 1.25\rho\rms{SW}$, which was found to accurately identify the WCR through prior testing.

\section{Results}

\subsection{Dust yields}

% Dust production rate

\begin{table}
  \centering
  \begin{tabular}{llll}
  \hline
  Parameter & Mean & Maximum & Minimum \\ \hline
  $\dot{\text{M}}\rms{d}$ (\si{\solarmass\per\year}) & \num{7.68e-08} & \num{1.24e-06} & \num{1.30e-19} \\
  $\bar{a}$ (\si{\micro\metre}) & \num{1.32e-02} & \num{1.44e-02} & \num{5.45e-03} \\
  $\bar{z}$ & \num{3.97e-04} & \num{3.32e-03} & \num{1.60e-07} \\ \hline
  \end{tabular}
  \caption[Advected scalar yields from WR140 simulation]{Advected scalar yields from the WR140 simulation.}
  \label{tab:paper-2-dust-rates}
\end{table}

The dust production were found to be consistent with the predictions made in our previous paper.
% Previous paper might be a bit of a whole thing, need to ask Julian how to reference to that
After an initial advection period lasting until $\Phi \approx 0.96$, the dust production rate rapidly increased as the stars approached periastron passage, peaking at $\phi = 1.005$.
After this maximum value, the 

(Fig. \ref{fig:wr140-dustproduction})

This is reflected in the overall dust mass of the simulation (Fig. \ref{fig:wr140-dustmass}), as well as in infrared observations of WR140, where the infrared emission from dust formation rapidly reaches a maximum value after periastron passage, and slowly relaxes to a minimum value. % Find citation for this, plot
This asymmetry in the time-dependent change in infrared luminosity implies the existence of several factors for suppression and encouragement of dust formation than the change in orbital separation distance.

It should be noted that due to the small size of the simulation, the dust mass in the system will reduce quickly, as dust advects off of the numerical grid.

\subsection{Instabilities}

% Instabilities

\begin{figure}
  \centering
  \includegraphics[width=\linewidth]{assets/wr140-dust_rate.pdf}
  \caption{}
  \label{fig:wr140-dustproduction}
\end{figure}

\begin{figure}
  \centering
  \includegraphics[width=\linewidth]{assets/wr140-m_dust.pdf}
  \caption{}
  \label{fig:wr140-dustmass}
\end{figure}


\begin{figure}
  \centering
  \includegraphics[width=\linewidth]{assets/visit-r1-min.png}
\end{figure}

\begin{figure*}
  \centering
  \includegraphics{assets/periastron-close-rhod.pdf}
\end{figure*}

\begin{figure*}
  \centering
  \includegraphics{assets/periastron-close-r0.pdf}
\end{figure*}

\begin{figure*}
  \centering
  \includegraphics{assets/periastron-rho.pdf}
\end{figure*}

\begin{figure*}
  \centering
  \includegraphics{assets/periastron-rhod.pdf}
\end{figure*}

\subsection{Influence of radial velocity on dust production}

% Note change in dust yields from previous paper, will have to reference this when in publishing

% Discuss change in relative velocity, will significantly impact, by as much as an order of magnitude? Write script to calculate change in chi?

% Inverse is true for secondary star, mitigated by it being less important, and velocity shift is less pronounced as orbitally dominant

% Velocity shear

% Whilst this will not dominate the dynamics of dust formation as a whole, this would explain asymmetry in dust formation rate curve



\begin{figure}
  \centering
  \includegraphics[width=\linewidth]{assets/radial-velocity/radial.pdf}
  \caption[Radial velocity]{Radial velocity as a function of the orbital phase for the WR and OB stars in the WR140 system relative to the barycentre. As periastron passage occurs, the sudden inversion from approaching to receding can alter the wind velocity of the WR star by as much as \SI{160}{\kilo\metre\per\second}. Whilst this discrepancy is $\sim 6\%$ of the WR wind velocity, this can significantly increase dust production if the stars are receding from each other.}
\end{figure}

\subsection{Other changes in velocity shear}

\section{Conclusion}

\section{Acknowledgements}


%%%%%%%%%%%%%%%%%%%% REFERENCES %%%%%%%%%%%%%%%%%%


\bibliographystyle{mnras}
\bibliography{references.bib} % if your bibtex file is called example.bib

% Don't change these lines
\bsp	% typesetting comment
\label{lastpage}
\end{document}

% End of mnras_template.tex
